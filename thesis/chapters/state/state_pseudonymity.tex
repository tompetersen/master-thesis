\section{Pseudonymisierung}

\label{sec_state_pseudonymity}

- A name or another bit string. Identifiers, which are generated using random data only, i.e., fully 
independent of the subject and related attribute values, do not contain side information on the 
subject they are attached to, whereas non-random identifiers may do. E.g., nicknames chosen by 
a user may contain information on heroes he admires; a sequence number may contain 
information on the time the pseudonym was issued; an e-mail address or phone number contains 
information how to reach the user.  \cite{pfitzmann2010}

- evtl. Common Criteria

- VANET

- TIMSI

%pfitzmann2001 - Abschnitt 12
Der Begriff der Pseudonymisierung beschreibt die Benutzung von Pseudonymen zur Identifizierung von Subjekten. Pseudonymisierung sagt dabei erst einmal lediglich etwas über die Verwendung eines Verfahrens aus, jedoch nichts über die daraus entstehenden Auswirkungen auf die Identifizierbarkeit eines Subjekts oder die Zurechenbarkeit bestimmter Aktionen. Hierfür spielen weitere Eigenschaften von Pseudonymen wie die folgenden eine Rolle:
\begin{itemize}
  \item garantierte Eindeutigkeit von Pseudonymen
  \item Möglichkeit von Pseudonymänderungen
  \item begrenzt häufige Verwendung von Pseudonymen 
  \item zeitlich begrenzte Verwendung von Pseudonymen
  \item Art der Pseudonymserstellung
\end{itemize}

Die Ausprägungen dieser Eigenschaften werden auch im Rahmen dieser Arbeit für das umzusetzende System zu bewerten sein.