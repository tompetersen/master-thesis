\section{Schwellwertschemata}

\begin{itemize}
  \item \textbf{Grundlagen} Von Shamirs Secret Sharing bis heute
  \item \textbf{Konkretes Verfahren} Im Detail erläutern - vmtl. Desmedt auf Basis von ElGamal
  \item Zusätzlich Erweiterungen dieses Verfahrens wie verteilte Schlüsselgenerierung.
\end{itemize}

%- Desmedt, Frankel: ElGamal \cite{DesmedtFrankel1990}
%- setzt zentralen "Dealer" voraus
%- Pedersen und verbesserte Variante 
%- Andere Möglichkeiten: Paillier, RSA, ...

\label{sec_state_threshold}

\subsection{Schemata}

Ein solches System, das auf dem ElGamal-Algorithmus und damit dem Diskreten-Logarithmus-Problem basiert, veröffentlichten die Autoren in \cite{DesmedtFrankel1990}. \todo{Details} Dieser Ansatz setzt in der \textit{Setup}-Phase auf eine zentrale vertrauenswürdige Stelle zur Erzeugung der Schlüssel und \textit{Shares}. In \cite{pedersen1991} stellt der Autor basierend auf diesen Ergebnissen ein Verfahren vor, das bei der Schlüsselgenerierung ohne eine vertrauenswürdige Instanz auskommt. Dieses Verfahren wird in \cite{gennaro1999} noch einmal verbessert.\\
Basierend auf dem jetzigen Recherchestand würde sich diese Kombination von Verfahren gut für den angestrebten Anwendungszweck eignen. Konkrete offene Implementierungen wurden jedoch bisher nicht gefunden, so dass möglicherweise eine eigene Implementierung umgesetzt werden muss.

Neben diesem Verfahren gibt es noch weitere Ansätze basierend auf RSA \cite{desmedt1993, nguyen2005} oder dem Paillier-Kryptosystem \cite{paillier1999, damgard2001}, die jedoch deutlich komplexer zu sein scheinen. 

\subsection{Desmedt und Frankel: Threshold cryptosystems}

Das ursprüngliche Verfahren setzt eine vertrauenswürdige, zentrale Instanz zur Schlüsselgenerierung voraus. Das Verfahren lässt sich in 4 Phasen unterteilen: Generierung von Schlüssel und Shares, Verschlüsselung einer Nachricht, Generierung der partiellen Entschlüsselungen mithilfe der Shares, Kombination der partiellen Entschlüsselungen zur ursprünglichen Nachricht. \todo{? Beziehen auf basics}

% Setup: Generierung von Schlüssel und Shares

Die zentrale Instanz erstellt ein Schlüsselpaar analog zu dem Vorgehen im ElGamal-Verfahren. Für eine \todo{Ergänzen oder ElGamal in Basics aufnehmen - siehe auch Handbook of applied cryptography}

Der private Schlüssel \(a\) wird dann mittels Shamir's Secret Sharing (siehe Abschnitt \ref{sec_basics_threshold_shamir}) in \(n\) Shares \(x_i, y_i\) für \(i \in {1, \dots, n}\) zerlegt, die dann an die entsprechenden Besitzer verteilt werden. Nachfolgend wird die zentrale Instanz nicht mehr benötigt und kann alle verbliebenen Werte im Speicher löschen.

% Verschlüsselung

Die Verschlüsselung einer Nachricht \(m\) funktioniert analog zum ursprünglichen ElGamal-Verfahren. Es wird ein zufälliger Wert \(r < q\) mit \(ggT(r,q) = 1\) generiert. Der Schlüsseltext ergibt sich als \((g^k, M * g^{a*r})\), wobei \(g^a\) den bereits generierten öffentlichen Schlüssel darstellt.

% Generierung der partiellen Entschlüsselungen

Die partiellen Verschlüsselungen 

\subsection{Dezentrale Schlüsselgenerierung}