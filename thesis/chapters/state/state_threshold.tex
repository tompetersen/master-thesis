\section{Schwellwertschemata}

\begin{itemize}
  \item \textbf{Grundlagen} Von Shamirs Secret Sharing bis heute
  \item \textbf{Konkretes Verfahren} Im Detail erläutern - vmtl. Desmedt auf Basis von ElGamal
  \item Zusätzlich Erweiterungen dieses Verfahrens wie verteilte Schlüsselgenerierung.
\end{itemize}

%- Desmedt, Frankel: ElGamal \cite{DesmedtFrankel1990}
%- setzt zentralen "Dealer" voraus
%- Pedersen und verbesserte Variante 
%- Andere Möglichkeiten: Paillier, RSA, ...

\label{sec_state_threshold}

Ein solches System, das auf dem ElGamal-Algorithmus und damit dem Diskreten-Logarithmus-Problem basiert, veröffentlichten die Autoren in \cite{DesmedtFrankel1990}. \todo{Details} Dieser Ansatz setzt in der \textit{Setup}-Phase auf eine zentrale vertrauenswürdige Stelle zur Erzeugung der Schlüssel und \textit{Shares}. In \cite{pedersen1991} stellt der Autor basierend auf diesen Ergebnissen ein Verfahren vor, das bei der Schlüsselgenerierung ohne eine vertrauenswürdige Instanz auskommt. Dieses Verfahren wird in \cite{gennaro1999} noch einmal verbessert.\\
Basierend auf dem jetzigen Recherchestand würde sich diese Kombination von Verfahren gut für den angestrebten Anwendungszweck eignen. Konkrete offene Implementierungen wurden jedoch bisher nicht gefunden, so dass möglicherweise eine eigene Implementierung umgesetzt werden muss.

Neben diesem Verfahren gibt es noch weitere Ansätze basierend auf RSA \cite{desmedt1993, nguyen2005} oder dem Paillier-Kryptosystem \cite{paillier1999, damgard2001}, die jedoch deutlich komplexer zu sein scheinen. 

\todo{TBW}