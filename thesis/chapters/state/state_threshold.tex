\section{Schwellwertschemata}

\begin{itemize}
  \item \textbf{Grundlagen} Von Shamirs Secret Sharing bis heute
  \item \textbf{Konkretes Verfahren} Im Detail erläutern - vmtl. Desmedt auf Basis von ElGamal
  \item Zusätzlich Erweiterungen dieses Verfahrens wie verteilte Schlüsselgenerierung.
\end{itemize}

\label{sec_state_threshold}

\subsection{Schemata}

\todo{Erweitern um weitere Systeme}

Ein solches System, das auf dem ElGamal-Algorithmus und damit dem Diskreten-Logarithmus-Problem basiert, veröffentlichten die Autoren in \cite{DesmedtFrankel1990}. \todo{Details} Dieser Ansatz setzt in der \textit{Setup}-Phase auf eine zentrale vertrauenswürdige Stelle zur Erzeugung der Schlüssel und \textit{Shares}. In \cite{pedersen1991} stellt der Autor basierend auf diesen Ergebnissen ein Verfahren vor, das bei der Schlüsselgenerierung ohne eine vertrauenswürdige Instanz auskommt. Dieses Verfahren wird in \cite{gennaro1999} noch einmal verbessert.\\
Basierend auf dem jetzigen Recherchestand würde sich diese Kombination von Verfahren gut für den angestrebten Anwendungszweck eignen. Konkrete offene Implementierungen wurden jedoch bisher nicht gefunden, so dass möglicherweise eine eigene Implementierung umgesetzt werden muss.

Neben diesem Verfahren gibt es noch weitere Ansätze basierend auf RSA \cite{desmedt1993, nguyen2005} oder dem Paillier-Kryptosystem \cite{paillier1999, damgard2001}, die jedoch deutlich komplexer zu sein scheinen. 

\subsection{Umzusetzendes kryptographisches Schwellwertschema}

%- Desmedt und Frankel, aufbereitet auch in Katz und Boneh.

%- Verfahren basierend auf Shamir und ElGamal

%- Analog zu basics-threshold-formal (zentrale) lässt sich das Verfahren in 4 Phasen unterteilen

Das Verfahren, das in einem späteren Teil dieser Arbeit implementiert wird, wurde in \cite{DesmedtFrankel1990} vorgestellt. Aufbereitete Darstellungen lassen sich in \cite{katz2014} und \cite{boneh2016} finden. Das Verfahren beruht auf einer geschickten Kombination von Shamir's Secret Sharing (Abschnitt \ref{sec_basics_threshold_shamir}) und des ElGamal-Kryptosystems (Abschnitt \ref{sec_basics_threshold_elgamal}).

Im Folgenden wird das entwickelte Schema entsprechend den in Abschnitt \ref{sec_basics_threshold_thresholddecryption} aufgeführten Algorithmen eines Threshold-Public-Key-Decryption-Systems im Detail dargestellt.

\textbf{Algorithmus G: Schlüsselgenerierung}

In dem Verfahren wird für die Schlüsselgenerierung eine zentrale, vertrauenswürdige Instanz vorausgesetzt, die den öffentlichen Schlüssel und die später benötigten Shares des geheimen Schlüssels erzeugt und verteilt. 

Zur Erzeugung werden zwei Primzahlen \(p\) und \(q\) mit der Eigenschaft \(p = 2q + 1\) - bekannt als sichere Primzahl bzw. Sophie-Germain-Primzahl - benötigt. Weiterhin ist ein Generator der Untergruppe der Ordnung \(q\) von \(\mathbb{Z}_p^*\) notwendig.

Der (temporär erstellte) geheime Schlüssel \(a \in \mathbb{Z}_q\) wird analog zu der Schlüsselgenerierung im ElGamal-Verfahren zufällig gewählt. Aus ihm wird der öffentliche Schlüssel \(pk = g^a \mod p\) berechnet.\\
Der geheime Schlüssel wird anschließend analog zu Shamirs Secret Sharing in \(\mathbb{Z}_q\) in einzelne Shares \((x_i, y_i) = (x_i, q(x_i))\) aufgeteilt und diese an die Teilnehmer verteilt. Anschließend werden diese Werte gelöscht, so dass nur noch die Teilnehmer im Besitz ihrer Shares und damit in der Lage sind, Schlüsseltexte zu entschlüsseln.

\textbf{Algorithmus E: Verschlüsselung}

Anschließend kann ein Klartext mithilfe von \(pk\) analog zu dem ElGamal-Verfahren 
%(siehe Abschnitt \ref{sec_basics_threshold_elgamal}) 
verschlüsselt werden. So erhält man \((v,c) = (g^k, m \cdot g^{ak})\) für ein durch den Sender zufällig gewähltes \(k \in \mathbb{Z}_q\).

\textbf{Algorithmus D: Partielle Entschlüsselung}

Jeder Besitzer eines Shares \((x_i, y_i)\) kann nun für den zu entschlüsselnden Schlüsseltext \((v,c)\) seine partielle Entschlüsselung \((x_i, v^{y_i})\) berechnen und diese an eine zentrale Instanz, den Combiner, senden. Empfängt dieser mindestens \(t\) partielle Entschlüsselungen\footnote{
  Zur Erinnerung: \(t\) beschreibt die Mindestzahl zur Entschlüsselung benötigter Shares des Schwellwertschemas.
}, so kann er den Klartext wiederherstellen.

\textbf{Algorithmus C: Kombination}

Hierzu berechnet der Combiner die Lagrange-Koeffizienten \(\lambda_i \in \mathbb{Z}_q\) wie in Shamir's Secret Sharing beschrieben\footnote{
  In diesem Abschnitt gilt \(i \in C\). \(C\) stellt dabei die Menge der Indizes der beteiligten Sharebesitzer dar. Es gilt also \(C \subseteq \{1, \dots, n\}\) und \(| C | \ge t\).
}. Anschließend kann
 
\[g^{ak} = \prod_{i=1}^k (v^{y_i})^{\lambda_i}\]

berechnet werden. Dies funktioniert, da 

\[
\prod_{i=1}^k (v^{y_i})^{\lambda_i} = 
\prod_{i=1}^k (g^k)^{y_i \cdot \lambda_i} = 
(g^k)^{\sum_{i=1}^{k} y_i \cdot \lambda_i} \overset{(*)}{=}
(g^k)^a
\]

gilt. Der letzte Schritt \((*)\) folgt direkt aus dem zugrundeliegenden Secret Sharing und ist in dieser Form bereits in Abschnitt \ref{sec_basics_threshold_shamir} zu finden.

Anschließend kann der Klartext als \(m = c \cdot (g^{ak})^{(-1)}\) wiederhergestellt werden. 



Ein Nachteil bei diesem Verfahren ist, dass für die Generierung des geheimen Schlüssels und der daraus resultierenden Shares eine zentrale und vertrauenswürdige Instanz notwendig ist. \todo{Erweitern, wenn beschrieben}
