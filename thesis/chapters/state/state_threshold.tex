\section{Schwellwertschemata}

\label{sec_state_threshold}

%\subsection{Schemata}

% - RSA signing/decryption \cite{frankel1997proactive} proactive
% - RSA signing/decryption \cite{gennaro1996robust} robust efficient
% - RSA signign/decryption \cite{rabin1998simplified} robust proactive
% - RSA signing \cite{nguyen2005}
% - RSA signing \cite{shoup2000practical}
% - Paillier encryption \cite{damgard2001, fouque2000sharing} -> homomorphic for electronic voting
% - DSS signing \cite{gennaro1996robustdss}
% - Schnorr \cite{stinson2001provably}



Aufbauend auf den Ideen von Shamir und Blakley und den ersten Ideen zu kryptographischen Schwellwertschemata wurden für verschiedene Algorithmen und Anwendungsfälle Schemata mit unterschiedlichen Eigenschaften entwickelt.

\subsection{Übersicht}

Eine Vielzahl von Veröffentlichungen behandlen das Problem der verteilten Erstellung von Signaturen: Die in \cite{shoup2000practical} entwickelte Lösung basiert auf dem RSA-Verfahren, \cite{gennaro1996robustdss} erweitert den DSS-Standard um ein Schwellwertschema und \cite{stinson2001provably} entwickelt ein Schema zur verteilten Signatur mittels Schnorr-Signaturen.\\
Weitere Forschungen haben sich mit der Entwicklung von RSA-basierten Schwellwertschemata zur verteilten Entschlüsselung beschäftigt, die im Kontext dieser Arbeit genutzt werden \cite{frankel1997proactive, gennaro1996robust, rabin1998simplified}. \\
Ein zusätzliches Verfahren, das im Zusammenhang mit verteilter Entschlüsselung Aufmerksamkeit erfuhr, ist das Paillier-Kryptosystem. In \cite{damgard2001} und \cite{fouque2000sharing} entwickelten die Autoren auf diesem System basierte Schwellwertschemata, die insbesondere durch ihre homomorphe Eigenschaft hervorstechen und dadurch im Bereich der elektronischen Wahlsysteme genutzt werden können.\\
Einen Überblick über weitere Veröffentlichungen in diesem Bereich bieten beispielsweise \cite{desmedt1997some}, \cite{gemmell1997} und \cite{desmedt1993}.

\subsection{ElGamal-basiertes Schwellwertschema}

\label{sec_state_threshold_scheme}

Ein Verfahren zur \textit{Threshold Decryption}, das auf auf einer geschickten Kombination von Shamir's Secret Sharing (Abschnitt \ref{sec_basics_threshold_shamir}) und des ElGamal-Kryptosystems (Abschnitt \ref{sec_basics_threshold_elgamal}) basiert, veröffentlichten die Autoren in \cite{DesmedtFrankel1990}. Aufbereitete Darstellungen lassen sich in \cite{katz2014} und \cite{boneh2016} finden. \\
Es ist eines der ersten veröffentlichten Schwellwertschemas und erfuhr dadurch viel Beachtung und entsprechend viele aufbauende Arbeiten, die Verbesserungen vorschlugen. Durch die hinterliegende Mathematik bietet das Schema einfachere Umsetzbarkeit (auch von Erweiterungen wie dezentraler Schlüsselgenerierung) gegenüber RSA\footnote{
  Das ElGamal-Verfahren nutzt zur Berechnung eine öffentlich bekannter Ordnung (sie ist Teil des öffenltichen Schlüssels). Im Gegensatz dazu werden Berechnungen bei RSA in \(\Phi(n)\) ausgeführt, das jedoch nicht öffentlich vorliegen darf \cite{nguyen2005}.
}. Dies gilt ebenso gegenüber den Paillier-basierten Schemata -- deren homomorphe Eigenschaften in dieser Arbeit nicht benötigt werden. Aus diesen Gründen fiel die Wahl des in dieser Arbeit umzusetzenden Schemas auf das genannte Verfahren.\\
Der Rest dieses Abschnitts stellt das Verfahren nun entsprechend den in Abschnitt \ref{sec_basics_threshold_thresholddecryption} aufgeführten Algorithmen eines Threshold-Public-Key-Decryption-Systems im Detail vor.

%\subsection{Umzusetzendes kryptographisches Schwellwertschema}

%- Desmedt und Frankel, aufbereitet auch in Katz und Boneh.

%- Verfahren basierend auf Shamir und ElGamal

%- Analog zu basics-threshold-formal (zentrale) lässt sich das Verfahren in 4 Phasen unterteilen

\textbf{Algorithmus G: Schlüsselgenerierung}

In dem Verfahren wird für die Schlüsselgenerierung eine zentrale, vertrauenswürdige Instanz vorausgesetzt, die den öffentlichen Schlüssel und die später benötigten Shares des geheimen Schlüssels erzeugt und verteilt. 

Zur Erzeugung werden zwei Primzahlen \(p\) und \(q\) mit der Eigenschaft \(p = 2q + 1\) - bekannt als sichere Primzahl bzw. Sophie-Germain-Primzahl - benötigt. Weiterhin ist ein Generator der Untergruppe der Ordnung \(q\) von \(\mathbb{Z}_p^*\) notwendig.

Der (temporär erstellte) geheime Schlüssel \(a \in \mathbb{Z}_q\) wird analog zu der Schlüsselgenerierung im ElGamal-Verfahren zufällig gewählt. Aus ihm wird der öffentliche Schlüssel \(pk = g^a \mod p\) berechnet.\\
Der geheime Schlüssel wird anschließend analog zu Shamirs Secret Sharing in \(\mathbb{Z}_q\) in einzelne Shares \((x_i, y_i) = (x_i, q(x_i))\) aufgeteilt und diese an die Teilnehmer verteilt. Anschließend werden diese Werte gelöscht, so dass nur noch die Teilnehmer im Besitz ihrer Shares und damit in der Lage sind, Schlüsseltexte zu entschlüsseln.

\textbf{Algorithmus E: Verschlüsselung}

Anschließend kann ein Klartext mithilfe von \(pk\) analog zu dem ElGamal-Verfahren 
%(siehe Abschnitt \ref{sec_basics_threshold_elgamal}) 
verschlüsselt werden. So erhält man \((v,c) = (g^k, m \cdot g^{ak})\) für ein durch den Sender zufällig gewähltes \(k \in \mathbb{Z}_q\).

\textbf{Algorithmus D: Partielle Entschlüsselung}

Jeder Besitzer eines Shares \((x_i, y_i)\) kann nun für den zu entschlüsselnden Schlüsseltext \((v,c)\) seine partielle Entschlüsselung \((x_i, v^{y_i})\) berechnen und diese an eine zentrale Instanz, den Combiner, senden. Empfängt dieser mindestens \(t\) partielle Entschlüsselungen\footnote{
  Zur Erinnerung: \(t\) beschreibt die Mindestzahl zur Entschlüsselung benötigter Shares des Schwellwertschemas.
}, so kann er den Klartext wiederherstellen.

\textbf{Algorithmus C: Kombination}

Hierzu berechnet der Combiner die Lagrange-Koeffizienten \(\lambda_i \in \mathbb{Z}_q\) wie in Shamir's Secret Sharing beschrieben\footnote{
  In diesem Abschnitt gilt \(i \in C\). \(C\) stellt dabei die Menge der Indizes der beteiligten Sharebesitzer dar. Es gilt also \(C \subseteq \{1, \dots, n\}\) und \(| C | \ge t\).
}. Anschließend kann
 
\[g^{ak} = \prod_{i=1}^k (v^{y_i})^{\lambda_i}\]

berechnet werden. Dies funktioniert, da 

\[
\prod_{i=1}^k (v^{y_i})^{\lambda_i} = 
\prod_{i=1}^k (g^k)^{y_i \cdot \lambda_i} = 
(g^k)^{\sum_{i=1}^{k} y_i \cdot \lambda_i} \overset{(*)}{=}
(g^k)^a
\]

gilt. Der letzte Schritt \((*)\) folgt direkt aus dem zugrundeliegenden Secret-Sharing-Schema und ist in dieser Form bereits in Abschnitt \ref{sec_basics_threshold_shamir} zu finden.

Anschließend kann der Klartext als \(m = c \cdot (g^{ak})^{(-1)}\) wiederhergestellt werden. 

Ein Nachteil dieses Verfahrens in der Phase der Schlüsselgenerierung ist, dass für die Generierung des geheimen Schlüssels und der daraus resultierenden Shares eine zentrale und vertrauenswürdige Instanz notwendig ist.

\subsection{Verteilte Schlüsselgenerierung}

\todo{Schreiben}

\cite{pedersen1991}

\cite{gennaro1999}

\subsection{Existierende Implementierungen}

Auch nach umfangreicher Recherche ließ sich keine quelloffene, kryptographisch überprüfte und lizenzrechtlich nutzbare Bibliothek finden, die das gewünschte Schwellwertschema implementiert. Es gab verschiedene verwandte Lösungen wie Civitas\footnote{
  Civitas -- A secure voting system. http://www.cs.cornell.edu/projects/civitas/
} oder Helios\footnote{
  Helios Voting. https://heliosvoting.org/
}, die jedoch alle eng mit dem Anwendungskontext der elektronischen Wahl verknüpft waren und dadurch andere Anforderungen erfüllten als sie für diese Arbeit erforderlich sind. Aus diesem Grund wird das beschriebene Schwellwertschema gezwungenermaßen in Teilen selbstständig implementiert.