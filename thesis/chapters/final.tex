\chapter{Fazit}

\label{cha_final}

%You generally cover three things in the Conclusions section, and each of these usually merits a separate subsection:
%1. Conclusions
%2. Summary of Contributions
%3. Future Research

%Conclusions are not a rambling summary of the thesis: they are short, concise statements of the inferences that you have made because of your work. It helps to organize these as short numbered paragraphs, ordered from most to least important. All conclusions should be directly related to the research question stated in Section 4. Examples:
  %1. The problem stated in Section 4 has been solved: as shown in Sections ? to ??, an algorithm capable of handling large-scale Zylon problems in reasonable time has been developed.
  %2. The principal mechanism needed in the improved Zylon algorithm is the Grooty mechanism.
  %3. Etc.

%The Summary of Contributions will be much sought and carefully read by the examiners. Here you list the contributions of new knowledge that your thesis makes. Of course, the thesis itself must substantiate any claims made here. There is often some overlap with the Conclusions, but that's okay. Concise numbered paragraphs are again best. Organize from most to least important. Examples:
  %1. Developed a much quicker algorithm for large-scale Zylon problems.
  %2. Demonstrated the first use of the Grooty mechanism for Zylon calculations.
  %3. Etc.

%The Future Research subsection is included so that researchers picking up this work in future have the benefit of the ideas that you generated while you were working on the project. Again, concise numbered paragraphs are usually best. 

\section{Conclusion}

In dieser Arbeit ...

\section{Contributions}

- Zusammenarbeit von Pseudonymisierung + Schwellwertschema
- Beleuchtung verschiedener Ansätze für das Suchproblem (Anwendung von Searchable Encryption für ...)
- Prototypische Implementation
- Implementation Schwellwertschema 

\section{Future work}

- Implementation:
  - Verteilte Schlüsselgenerierung
  - Performance (Elliptic Curve Crypto), ...
  - Cryptographic Review
  - Access structures (irgendwo noch erwähnen!! State Threshold?)
- Parameterwahl für situationsabhängige Pseudonymisierung? (Evtl. in state noch ergänzen...) -> Forschungsarbeit notwendig
- Auswirkungen für verteilte Systeme (Schlüssel, SearchableEncryption, ...)
