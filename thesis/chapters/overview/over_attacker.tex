\section{Zugrundeliegendes Angreifermodell}

\label{subsec_impl_requirements_attackermodel}

% Das Angreifermodell definiert die	maximal	berücksichtigte	Stärke eines Angreifers, gegen den ein	Schutzmechanismus	gerade noch wirkt.	
%Es beschreibt:
%–  Rollen des Angreifers (Außenstehender, Benutzer, Betreiber, Wartungsdienst, Produzent, Entwerfer …), auch kombiniert 
%–  Verbreitung des Angreifers (Stellen im System, an denen der Angreifer Informationen gewinnen oder Systemzustände verändern kann) 
%–  Verhalten des Angreifers 
%  •  passiv / aktiv,  beobachtend / verändernd 
%–  Rechenkapazität des Angreifers 
%  •  unbeschränkt: informationstheoretisch 
%  •  beschränkt: komplexitätstheoretisch 

Im Kontext dieser Arbeit, die sich mit der datenschutzfreundlichen Speicherung von Überwachungsdaten beschäftigt, muss zuerst folgende Vorüberlegung getroffen werden: Logdaten erreichen das verwendete SIEM-System abhängig von den verwendeten Protokollen im allgemeinen nicht-pseudonymisiert und oftmals weder verschlüsselt noch mit geschützer Integrität über das Netzwerk. \todo{Angriffsmöglichkeiten erläutern: Mitschneiden aller Daten und damit  Aufdecken von Pseudonymen, ...}
Diese Angriffsmöglichkeiten zu verhindern, ist ausdrücklich kein Ziel dieser Arbeit. Daher bezieht sich das Angreifermodell auf die Bearbeitung und Speicherung von Logdaten, erst sobald sie das zu entwickelnde System erreichen, jedoch nicht vorher.

Das Ziel des Systems bezogen auf seine Sicherheit lässt sich folgendermaßen definieren: Das Pseudonym eines Nutzers erlaubt (ohne Anwendung von Hintergrundwissen) keinen Rückschluss auf die Identität eines Nutzers. Erst die Kollaboration berechtigter Akteure ermöglicht das Aufdecken eines Pseudonyms.\todo{? außerdem erweitern um aktive Angreifer?}

Ein Angreifermodell beschreibt die maximale Stärke eines Angreifers im Bezug auf verschiedene Faktoren, gegen die ein System abgesichert ist. Enthalten sind die Rolle eines Angreifers, seine Verbreitung im System, aktives oder passives Verhalten und die Rechenkapazität, die der Angreifer zum Überwinden der eingesetzten Schutzmaßnahmen aufbringen kann. Es bildet die Basis für alle Folgeüberlegungen im Bezug auf die Sicherheit des zu entwickelnden Systems.

Für das Angreifermodell werden folgende Annahmen getroffen: Bei dem Angreifer kann es sich um einen Außenstehenden, um einen Berechtigten mit Zugriff auf das SIEM-System oder sogar um einen Administrator mit physischem Zugriff auf Rechner im Netz handeln. Wie bereits beschrieben, wird die ungesicherte Übertragung nicht-pseudonymisierter Daten zu dem System nicht betrachtet. Insofern wird von einem Angreifer ausgegangen, der erst Zugriff auf das zu entwickelnde System oder das SIEM-System beseitzt oder erlangt. Das zu entwickelnde System soll in der Lage sein, sich auch gegen aktive Angreifer zur Wehr zu setzen. \todo{Denial of Service etc?}
Bezogen auf die verfügbare Rechenleistung des Angreifers sollen verbreitete und nach heutigem Wissensstand für sicher befundene kryptographische Algorithmen als nicht mit vertretbarem Aufwand zu brechen angesehen werden. Es handelt sich daher um die Annahme von komplexitätstheoretischer Sicherheit.
