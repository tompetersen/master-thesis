\section{Angreifermodell}

\label{subsec_impl_requirements_attackermodel}

% Das Angreifermodell definiert die	maximal	berücksichtigte	Stärke eines Angreifers, gegen den ein	Schutzmechanismus	gerade noch wirkt.	
%Es beschreibt:
%–  Rollen des Angreifers (Außenstehender, Benutzer, Betreiber, Wartungsdienst, Produzent, Entwerfer …), auch kombiniert 
%–  Verbreitung des Angreifers (Stellen im System, an denen der Angreifer Informationen gewinnen oder Systemzustände verändern kann) 
%–  Verhalten des Angreifers 
%  •  passiv / aktiv,  beobachtend / verändernd 
%–  Rechenkapazität des Angreifers 
%  •  unbeschränkt: informationstheoretisch 
%  •  beschränkt: komplexitätstheoretisch 


Das Ziel des Systems bezogen auf seine Sicherheit lässt sich folgendermaßen definieren: Das Pseudonym eines Nutzers erlaubt (ohne Anwendung von Hintergrundwissen) keinen Rückschluss auf die Identität eines Nutzers. Erst die Kollaboration berechtigter Akteure ermöglicht das Aufdecken eines Pseudonyms. Aus diesem Grund soll sich auch das nachfolgend aufgestellte Angreifermodell auf dieses Ziel fokussieren. Andere Angriffsarten beispielsweise auf die Verfügbarkeit des Systems werden daher nicht betrachtet.

Ein Angreifermodell beschreibt die maximale Stärke eines Angreifers im Bezug auf verschiedene Faktoren, gegen die ein System abgesichert ist. Enthalten sind die Rolle eines Angreifers, seine Verbreitung im System, aktives/passives und beobachtendes/veränderndes Verhalten und die Rechenkapazität, die der Angreifer zum Überwinden der eingesetzten Schutzmaßnahmen aufbringen kann. \cite{baumann2014kryptographische} 

Bezogen auf die verfügbare Rechenleistung des Angreifers sollen verbreitete und nach heutigem Wissensstand für sicher befundene kryptographische Algorithmen als nicht mit vertretbarem Aufwand zu brechen angesehen werden. Es handelt sich daher um die Annahme von komplexitätstheoretischer Sicherheit. Das in Abschnitt \ref{sec_basics_threshold_elgamal} erwähnte Diskrete-Logarithmus-Problem wird also beispielsweise für ausreichend große Primzahlen als nicht mit vertretbarem Aufwand zu brechen angesehen.

Im Bezug auf die Verbreitung eines Angreifers muss zuerst folgende Vorüberlegung getroffen werden: 
Logdaten erreichen das verwendete SIEM-System abhängig von den verwendeten Protokollen im allgemeinen nicht-pseudonymisiert und oftmals weder verschlüsselt noch mit Schutz ihrer Integrität über das Netzwerk. Hierdurch könnte ein Angreifer bereits vor dem Eintreffen der Daten im Proxy passiv alle Daten mitlesen und die anschließend stattfindende Pseudonymisierung würde nichts an dem gewonnen Wissen ändern können.
Es ist kein Ziel dieser Arbeit, diese Angriffsmöglichkeiten zu verhindern, da sie sich mit der datenschutzfreundlichen \textbf{Speicherung} von Überwachungsdaten beschäftigt. Daher werden bei der Verbreitung eines Angreifers die Datenquellen und Übertragungswege zum Log-Proxy nicht betrachtet.

Mit dieser Einschränkung ergeben sich verschiedene Rollen und darauf basierend Verbreitungen, die ein Angreifer annehmen kann:

\begin{description}
  \item[Außenstehender] Als Außenstehender wird hier jeder Akteur verstanden, der keinen legitimen Zugriff auf Teile des Systems besitzt. Gemeint sind also genauso Mitarbeiter in einem Unternehmensnetzwerk, in dem das System genutzt wird, wie externe Angreifer. Sind die  Zugriffsmechanismen im Pseudonym-Service korrekt umgesetzt, bleibt Außenstehenden nur das passive Beobachten von Nachrichten. Durch den Einsatz der in Abschnitt \ref{sec_over_architecture} erwähnten notwendigen Transportverschlüsselung erfahren passive Angreifer jedoch keine brauchbaren Informationen. Diese verhindert auch das Verändern der Nachrichten auf der Übertragungsstrecke durch aktive Angreifer.
  
  \item[Benutzer mit SIEM-Zugriff] Ein Benutzer, der Zugriff auf das SIEM-System besitzt, sieht nur die pseudonymisiert gespeicherten Logdaten. Hieraus erfährt er erst einmal nichts über den Benutzer hinter dem Pseudonym. Durch die Anwendung von Hintergrundwissen und die Verknüpfung von Datenbankeinträgen kann er bestimmte Pseudonyme eventuell aufdecken -- ein Angriff, der nicht zu verhindern ist, durch regelmäßige Pseudonymwechsel jedoch zumindest in seiner Reichweite beschränkt werden kann.
  
  \item[Benutzer mit Recht auf Pseudonymaufdeckung] Legitimierte Benutzer können Anfragen zur Aufdeckung eines Pseudonyms stellen. Durch den Einsatz des kryptographischen Schwellwertschemas führt jedoch erst die Kollaboration von ausreichend Share-Besitzern zur wirklichen Aufdeckung. Vorher erfährt der Benutzer nichts über den Pseudonymhalter.
  
  \item[Share-Besitzer] Besitzer eines Shares erhalten für aufzudeckende Pseudonymzuordnungen die verschlüsselten Daten und berechnen aus ihrem Share und dem Schlüsseltext eine partielle Entschlüsselung. Ausgehend von den Eigenschaften des kryptographischen Schwellwertschemas erfahren sie hieraus jedoch nichts über das verschlüsselte Datum, solange nicht eine Mindestanzahl \(t\) an partiellen Entschlüsselungen vorliegt. Erst eine Kollaboration von mindestens \(t\) als aktive Angreifer handelnden Share-Besitzern kann so unberechtigt Pseudonyme aufdecken, wenn sie zusätzlich Zugriff auf die Datenbank des Pseudonym-Service erlangen. \\
  Eine Kollaboration von mindestens \(n-t+1\) bösartigen Share-Besitzern könnte auch dazu führen, dass die Aufdeckung eines Pseudonyms durch das Senden fehlerhafter partieller Entschlüsselungen fehlschlägt. Die sinnvollen Aufteilung der Shares und damit die Modellierung von verteiltem Vertrauen spielt also in dem System eine wichtige Rolle.\\
  Zusätzlich könnten im Falle der verteilten Schlüsselgenerierung bösartige Share-Besitzer beispielsweise durch das Senden falscher Daten versuchen, die Generierung zu stören und damit das anschließende Aufdecken von Pseudonymen zu verhindern. Dieser Angriff muss bei der Schlüsselgenerierung durch das verwendete Schema verhindert werden.
  
  \item[Administrator mit Proxy-Zugriff] Der Zugriff auf den Proxy, an dem die Logdaten im Klartext eintreffen und pseudonymisiert werden, erlaubt die Zuordnung von Pseudonymen zu Benutzern. Hier muss der Zugriff nach der Initialisierung eingeschränkt werden und soweit wie möglich sichergestellt werden, dass auch zugriffsberechtigte Benutzer den Zugriff nicht ausnutzen (z.B. durch Mehraugenprinzip oder zumindest Protokollierung der Handlungen).
  
  \item[Administrator mit Pseudonym-Service-Zugriff] Der Administrator des Pseudonym-Service hat Zugriff auf die Datenbank der verschlüsselten Pseudonymzuordnungen. Durch den Einsatz des Schwellwertschemas und die Verschlüsselung der Logdaten-Pseudonym-Zuordnung bereits im Proxy erfährt er (kein Hintergrundwissen vorausgesetzt) nichts über die Pseudonymhalter. Besondere Bedeutung kommt der Schlüsselgenerierung im Pseudonym-Service im Falle der zentralen Schlüsselgenerierung zu. Gelingt es dem Administrator während der Schlüsselgenerierung in den Besitz des temporär erzeugten geheimen Schlüssels oder mindestens \(t\) Shares zu kommen, so kann er jederzeit in der Datenbank abgelegte Pseudonymzuordnungen entschlüsseln, ohne dass andere Benutzer dies mitbekommen. Aus diesem Grund ist die verteilte Schlüsselgenerierung zu bevorzugen.
\end{description}