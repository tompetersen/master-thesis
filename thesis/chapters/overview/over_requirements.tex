\section{Anforderungen}

\label{sec_impl_requirements}

Neben den primären Anforderungen, die sich direkt aus der Funktionsbeschreibung des Systems und dem Zusammenspiel der enthaltenen Verfahren ergeben, sollte das System noch weitere Eigenschaften wie beispielsweise die Erweiterbarkeit um weitere Dateschutztechniken erfüllen. Alle diese Anforderungen sollen im folgenden Abschnitt aufgestellt und näher erläutert werden.

\subsection{Anforderungen an die Integration in das SIEM-System}

\label{subsec_impl_requirements_ossimintegration}

\todo{Diese Eigenschaften hier oder erst Entwurf und hier nur allgemein?}

Um zu beurteilen, an welcher Stelle in den Datenfluss der Logdaten in dem verwendeten SIEM-System  eingegriffen wird, um die Daten zu verändern, müssen Vor- bzw. Nachteile der verschiedenen Möglichkeiten gegeneinander abgewogen werden. Folgende Eigenschaften einer Möglichkeit sollten betrachtet werden:

\begin{itemize}

  \item \textbf{Veränderung des SIEM-Systems: } Muss das eingesetzte SIEM-System für die Umsetzung der Lösung angepasst werden? Dies wäre im Hinblick auf zukünftige Updates, die das SIEM-System durch seinen Entwickler erfährt, nicht wünschenswert, da jedes dieser Updates dafür sorgen könnte, dass die umgesetzte Lösung angepasst werden muss. Weiterhin würde dieser Ansatz ein SIEM-System erfordern, dass entweder quelloffen vorliegt und verändert werden darf oder die gewünschte Verhaltensänderung zumindest durch Erweiterungen zulässt. 
  
  \item \textbf{Nicht-pseudonymisierte Daten im SIEM-System: } Um das Ziel der Arbeit -- die Pseudonymisierung, die nur durch Kollaboration aufgedeckt werden kann -- zu erreichen, muss sichergestellt sein, dass Logdaten nirgendwo in nicht pseudonymisierter Form vorliegen. Da insbesondere das zukünftige Verhalten des SIEM-Systems nicht beeinflusst werden kann, wäre es wünschenswert, dass die Logdaten bereits in pseudonymisierter Form das SIEM-System erreichen.\\
  Die Relevanz dieser Eigenschaft lässt sich am Beispiel des später in dieser Arbeit genutzten SIEM-Systems OSSIM erkennen: Wird das Syslog-Protokoll genutzt, um Logdaten in OSSIM aufzunehmen, so werden die Einträge erst in einer Logdatei abgelegt und von dort aus geparst, normalisiert und in der Datenbank gespeichert. Das Datum verbleibt in der Logdatei. Kommen die Daten in nicht-pseudonymisierter Form in dem OSSIM-Sensor an, so muss sichergestellt werden, dass verarbeitete Einträge gelöscht oder verändert werden.
  
  \item \textbf{Mehrfaches Parsen von Logdaten: } Durch das SIEM-System werden die Logdaten - wie in Abschnitt \ref{sec_basics_siem} beschrieben - geparst und normalisiert. Aus Performancegründen ist eine Lösung zu bevorzugen, die diesen Vorgang oder Teile davon nicht mehrfach voraussetzt.
  
  \item \textbf{Abhängigkeit von Besonderheiten des SIEM-Systems: } Einige SIEM-Systeme bieten Möglichkeite der verteilten Installation oder andere spezifische Eigenschaften. Eine Lösung, die unabhängig von dem verwendeten SIEM-System funktioniert, ist zu bevorzugen, da sie universell einsetzbar ist.    
  
%  \item \textbf{Manipulierbarkeit auf den Übertragungswegen(?): }

\end{itemize}

\subsection{Anforderungen an die Pseudonymisierung}

\label{subsec_impl_requirements_pseudonymity}

%- Lang genug für geringe Kollisionswsk.
%- Eindeutig
%- Durchsuchbar (mim Hinblick auf threshold)
%- Anwendungsfallabhängige Parameter für Nutzzeit, ... (Rückblick auf Kapitel 3)

Die Pseudonymisierung muss es ermöglichen nach Aufdecken eines Eintrags wieder auf den ursprünglichen Dateninhalt schließen zu können. Daher müssen die Pseudonyme für die Zeit ihrer Speicherung eindeutig sein, d.h. es darf zu keiner gleichzeitigen Mehrfachverwendung von Pseudonymen kommen. 

Weiterhin muss es eine Möglichkeit beim Pseudonymisieren von Logeinträgen geben, zu überprüfen, ob für ein Datum bereits ein Pseudonym vergeben wurde. So kann sichergestellt werden, dass in einem bestimmten Zeitraum Logeinträge zu einer Person mit dem gleichen Pseudonym versehen werden, um über die Verknüpfung von Einträgen die gewünschte Erkennung von Insider-Angriffen erreichen zu können. \todo{Auf state-SE beziehen?}

Außerdem muss es eine Möglichkeit geben, die Parameter der Pseudonymisierung wie den Zeitraum ihrer Verwendung (vergleiche Abschnitt \ref{sec_state_pseudonymity}) konfigurierbar zu machen.

\todo{Mit Blick auf state-pseudo nochmal überprüfen und anpassen}

\subsection{Anforderungen im Bezug auf den Einsatz eines kryptographischen Schwellwertschemas}

\label{subsec_impl_requirements_threshold}

%- Verteiltes Modell 
%- Kommunikation
%- Schlüsselmanagement
%- ...

\todo{Anpassen nach state - Threshold-Abschnitt?}

Der Einsatz eines kryptographischen Schwellwertschemas setzt eine verteilte Anwendung voraus, die den Zugriff für die Pseudonymisierungskomponente sowie für die bei der Entschlüsselung eines Eintrags beteiligten Akteure bereitstellt. Die für das Schwellwertschema nötigen Parameter \(t\) und \(n\) sowie die beteiligten Akteure müssen anpassbar bzw auswählbar sein.

In der Phase der Schlüsselgenerierung muss das System die Kommunikation und Koordination aller Beteiligten unterstützen. Die hier erstellten Schlüssel und \textit{Shares} müssen an geeigneten Stelle sicher gespeichert und abrufbar sein.

Der für die Verschlüsselung erforderliche öffentliche Schlüssel muss so vorliegen, dass er bei der Verschlüsselung eines Pseudonym-Datensatzes genutzt werden kann.

Bei der Entschlüsselung eines Eintrags, also der Aufdeckung eines Pseudonyms, muss das System wiederum die beteiligten Akteure koordinieren und anschließend die Rolle des \textit{Combiners} übernehmen, so dass anschließend der entschlüsselte Datensatz zentral vorliegt.

\subsection{Benutzerinteraktion}

\label{subsec_impl_requirements_userinteraction}

%- Proxy: Konfiguration der Plugins

%- Pseudo-App: Statusanzeige angemeldeter Benutzer(Admin), Initialisierung der Schlüsselgenerierung nach Nutzerauswahl(Admin), Anlegen von Aufdeckanfragen, Statusanzeige von Aufdeckanfragen, Systemstatus

%- Einzelne Teilnehmer sollten Client-Anwendungen besitzen, um auf Anfragen reagieren zu können (Generation eigener Schlüssel, gemeinsame Schlüsselgenerierung, ... ) Konsolenanwendung? 

Die zu entwickelnde verteilte Anwendung wird an verschiedenen Stellen Benutzerinteraktion erfordern.

Das Konfigurieren des Systems zur Integration verschiedener Datenquellen und deren Beschreibung muss einem berechtigten Nutzer zugänglich gemacht werden. Ebenso sollte es im Hinblick auf die in der Aufgabenstellung geforderte Erweiterbarkeit im Bezug auf weitere Datenschutztechniken relativ leicht sein, diese Techniken im System nutzen zu können. 

Für pseudonymisierte Datensätze muss es berechtigten Benutzern ermöglicht werden, Anfragen zur Aufdeckung eines Pseudonyms zu stellen und sich über ihren Status informiert zu halten.

Für die Benutzung eines kryptographischen Schwellwertschemas sollte es einem Administrator des Systems ermöglicht werden, grundlegende Parameter des Systems wie die Schwellwertparameter und die beteiligten Nutzer auszuwählen sowie die Initialisierung des Schemas anzustoßen. \\
Die am Schwellwertschema beteiligten Nutzer müssen die Möglichkeit erhalten, eine Übersicht über sie betreffende Anfragen zur Aufdeckung eines Pseudonym-Datensatzes zu bekommen sowie einzelne Anfragen abzulehnen oder sich am Prozess des Aufdeckens mithilfe des Schwellwertschemas zu beteiligen. 

\subsection{Erweiterbarkeit um neue Datenquellen}

\label{subsec_impl_requirements_differentsources}

Das umzusetzende System sollte es ermöglichen, Daten aus verschiedenen Quellen und (abhängig vom gewählten Eingriffspunkt in OSSIM auch) in verschiedenen Formaten entgegenzunehmen und mithilfe der umgesetzten Datenschutztechniken verändern zu können. Dabei muss das Format der Logdaten grundsätzlich beibehalten werden, um die Behandlung der Daten in dem verwendeten SIEM-System weiterhin zu ermöglichen.

\subsection{Erweiterbarkeit um neue Datenschutztechniken}

\label{subsec_impl_requirements_plugins}

Neben der im Fokus dieser Arbeit stehenden Pseudonymisierung und dem Einsatz von kryptographischen Schwellwertschemata zum Schutz der Logdaten gibt es weitere Datenschutztechniken, die für den Anwendungsfall genutzt werden könnten (siehe Kapitel \ref{cha_alternatives}). Der umgesetzte Prototyp sollte leicht um diese Techniken erweiterbar sein, d.h. so gestaltet sein, dass andere Techniken ohne große Änderungen am System integriert und auf eingehende Logdaten angewendet werden können.

\todo{Performance?}

\subsection{Übersicht}

Zusammenfassend sollte ein System, wie es in dieser Arbeit angestrebt wird, also folgende Anforderungen erfüllen:

\begin{itemize}
  \item Geeignete Stelle zum Eingriff in den Datenfluss zwischen Logdatenquelle und SIEM-System
  \item Parameterabhängige Generierung eindeutiger, aber in gewissem Rahmen verknüpfbarer Pseudonyme
  \item Sicherer, verteilter Einsatz eines anpassbaren kryptographischen Schwellwertschemas
  \item Geeignete Benutzerinteraktion mit dem System an notwendigen Stellen
  \item Erweiterbarkeit um unbekannte Datenquellen
  \item Erweiterbarkeit um weitere Datenschutztechniken
\end{itemize}
