\chapter{Ergänzende und alternative Datenschutztechniken}

\label{cha_alternatives}

%\begin{itemize}
%  \item \textbf{Alternativen} Welche alternativen oder ergänzenden Vorgehensweisen zu Pseudonymisierung + kryptographisches Schwellwertschema gibt es und welche Eigenschaften, Vor- und Nachteile besitzen sie?
%  \item \textbf{Umsetzung} Wie könnten diese Alternativen im Prototypen umgesetzt werden?
%\end{itemize}

% Generalisierung (zb nur noch Abteilung betrachten)
% Löschung
% Rauschen hinzufügen zB Zeitstempel plus normalverteilten Zufallswert (Christian)
% ...
%

% Siehe auch 
% Niksefat et. al.: Privacy issues in intrusion detection systems: A taxonomy, survey and future directions
%
%- Hash functions
%- Bloom filters
%- Homomorphic encryption
%- Secure multiparty computation
%- Z-String
%- Concept hierarchy
%- Differential privacy
%- Other classical techniques: Removal, Perturbation (Adding noise), Shifting (for example all timestamps to preserve interval lengths but hide real datetime)



Der in dieser Arbeit verfolgte Ansatz der Pseudonymisierung unter Einsatz kryptographischer Schwellwertschemata ist nicht für alle Arten von Logdaten sinnvoll. Sollen beispielsweise Zeitstempel verändert werden, um keine direkten Rückschlüsse auf eine Person durch Kombination mit typischen Verhaltensmustern zu ermöglichen, auf der anderen Seite jedoch zumindest grobe Erkenntnisse aus dem Zeitstempel für die Anomalieerkennung genutzt werden, so kann eine auf Pseudonymisierung basierende Lösung dies nicht leisten. 

Daher werden in diesem Abschnitt ergänzende Techniken zum Erhalt der Privatspäre eines Arbeitnehmers bei der Speicherung von Logdaten dargestellt und erläutert, wie diese in den entwickelten Prototyp eingebunden werden können. Natürlich können auch mehrere kombinierte Techniken für einzelne Logdaten, die aus mehreren Feldern bestehen, sinnvoll sein. Mit Zeitstempeln versehene Logdaten eines Türschließsystems, die die Benutzerkennungen der Mitarbeiter enthalten, könnten so beispielsweise durch Pseudonymisierung der Benutzerkennungen und Verrauschung der Zeitstempel geschützt werden.

Eine Übersicht über weitere in sehr speziellen Einsatzgebieten zu nutzende Datenschutztechniken wie Bloomfilter oder der Einsatz homomorpher Verschlüsselung sind in \cite{niksefat2017privacy} zu finden.

\section{Unterdrückung} % Suppression

Als ergänzende Maßnahme für Datenfelder, die für die Anomalieerkennung nicht benötigt werden, aber Rückschlüsse auf den Benutzer zulassen, kommt die Unterdrückung in Frage. Hierbei wird der Wert des Feldes schlicht entfernt oder durch eine Konstante ersetzt. 

Ein Beispiel im Kontext dieser Arbeit ist die Unterdrückung von IP-Adressen eines durch einen Arbeitnehmer benutzten Rechners für Logdaten, in denen auch der Benutzername enthalten ist. Beim Einsatz von Pseudonymisierung könnte die Zuordnung von Benutzer zu Pseudonym erleichtert werden, wenn der Arbeitnehmer dem physischen Gerät zu einem Zeitpunkt zugeordnet werden kann und die IP-Adresse des Geräts noch in den Logdaten enthalten ist.

\section{Generalisierung}

Bei der Generalisierung wird der Feldinhalt durch einen Wert ersetzt, der das gleiche Konzept beschreibt, jedoch allgemeiner ist. Durch mehrfache Verallgemeinerung entstehen sogenannte Generalisierungshierarchien \todo{Beispiel, Diagramm?}, wobei die Stufe der höchsten Generalisierung hier gleichbedeutend mit der Unterdrückung ist, da jeder Wert durch den konstanten Wert der höchsten Generalisierungsstufe ersetzt wird.

Ein Beispiel im Unternehmenskontext dieser Arbeit ist die Generalisierung eines Mitarbeiters zu seiner Arbeitsgruppe oder Abteilung -- eine Information, die für die Anomalieerkennung ausreichend sein könnte, wenn es beispielsweise um Zugriffe auf Ressourcen geht, die für bestimmte Abteilungen üblich, für andere jedoch ungewöhnlich sind. Dieses könnte zusätzlich zur Pseudonymisierung ausgeführt werden, um der Anomalieerkennung zusätzliche Daten zur Verfügung zu stellen, ohne die Identität eines Nutzers direkt offenzulegen.

\section{Verrauschen} % Hinzufügen von Rauschen

Diese Maßnahme verändert den Wert eines Datenfeldes, indem diesem Feld Werte aus einer Wahrscheinlichkeitsverteilung hinzugefügt werden (statistisches Rauschen). 
Hierdurch lassen sich die Rückschlüsse auf einen Nutzer aus einem einzelnen Datensatz verringern, aber die Gesamtverteilung bleibt erhalten bzw. lässt sich leicht berechnen. 
So können zumindest Abweichungen von Durchschnittswerten zur Anomalieerkennung genutzt werden. 
Es wird jedoch eine ausreichend große Datenmenge benötigt. Alternativ lassen sich zumindest Aussagen über den Bereich treffen, in dem ein Wert sich befinden muss. Dies könnte beispielsweise beim Verrauschen von Ereigniszeitstempeln sinnvoll sein, bei dem zwar nicht auf den konkreten Zeitpunkt geschlossen werden kann, aber zumindest Aussagen darüber getroffen werden können, ob das Ereignis in einem üblichen Intervall, wie in den normalen Bürostunden, auftrat.

Die Maßnahme ist jedoch nur für bestimmte Felder bzw. Datenarten sinnvoll einsatzbar. Gegenbeispiele sind unter anderem Freitextfelder, wie Benutzernamen, oder Felder für Aufzählungstypen, wie Raumnummern, die sich mit Rauschen nicht sinnvoll verändern lassen. 

\section{Nutzung von Hashverfahren}

Neben der zufälligen Generierung von Pseudonymen, wie es in dieser Arbeit genutzt wird, ist auch die Nutzung von Hashwerten als Pseudonym für Daten denkbar. Dies würde die Verknüpfbarkeit von Logdaten ermöglichen, da für gleiche Daten der gleiche Hashwert berechnet wird. Durch den geschickten Einsatz von zusätzlichen zeitabhängig wechselnden Eingaben für das Hashverfahren (sogenannte \textit{Salts}) ließe sich auch die nötige Verknüpfbarkeit für die Anomalieerkennung gegenüber dem Schutz der Privatsphäre eines Benutzers abstimmen.\\
Auf der anderen Seite wäre der Einsatz von Hashverfahren bei einem kleinen Wertebereich für Eingaben wie Benutzernamen anfällig für Wörterbuchangriffe (vgl. Abschnitt \ref{sec_state_se}). Außerdem wären auch Rückschlüsse auf den Pseudonymhalter nicht ohne zusätzlichen Aufwand möglich.

Für Datenfelder, bei denen nur die Verknüpfbarkeit, jedoch nicht der ursprüngliche Wert, für die Anomalieerkennung entscheidend ist und deren Wertebereich ausreichend groß ist, können Hashverfahren sinnvoll sein. \todo{Beispiel}

\section{Vorgehen zur Integration}

Die Integration der vorgestellten Datenschutztechniken in den entwickelten Prototyp stellt kein Problem dar. Die jeweiligen Techniken können, wie in Abschnitt \ref{sec_integration_in_ossim_plugins} beschrieben, als Plugins entwickelt werden. Hierbei entsteht je nach Datenschutztechnik unterschiedlicher Aufwand.

Für die Generalisierung und das Verrauschen müssen beispielsweise eingabedatenabhängige Plugins entstehen. Die Generalisierung von Zeitstempeln (Generalisierung auf Minute, Stunde, ... oder selbst gewählte Zeitabschnitte) unterscheidet sich beispielsweise stark von der Generalisierung der Umgebung eines Mitarbeiters (unternehmensspezifische Generalisierung auf Arbeitsgruppe, Abteilung, ...). Die Unterdrückung oder Nutzung von Hashverfahren kann hingegen unabhängig von den Eingaben entwickelt werden.

Für Datenquellen können diese Plugins nun einzeln oder in Kombination, wie in Abschnitt \ref{sec_integration_in_ossim_datasource_config} beschrieben, genutzt werden. Durch diesen Ansatz lässt sich für jede Datenquelle abhängig von verwendeten Anomalieerkennungsverfahren eine gute Abwägung zwischen Nutzbarkeit der Daten und Schutz der Privatsphäre eines Arbeitnehmers schaffen.