\chapter{Alternativen}

\label{cha_alternatives}

%\begin{itemize}
%  \item \textbf{Alternativen} Welche alternativen oder ergänzenden Vorgehensweisen zu Pseudonymisierung + kryptographisches Schwellwertschema gibt es und welche Eigenschaften, Vor- und Nachteile besitzen sie?
%  \item \textbf{Umsetzung} Wie könnten diese Alternativen im Prototypen umgesetzt werden?
%\end{itemize}

% Generalisierung (zb nur noch Abteilung betrachten)
% Löschung
% Rauschen hinzufügen zB Zeitstempel plus normalverteilten Zufallswert (Christian)
% ...
%

% Siehe auch 
% Niksefat et. al.: Privacy issues in intrusion detection systems: A taxonomy, survey and future directions
%
%- Hash functions
%- Bloom filters
%- Homomorphic encryption
%- Secure multiparty computation
%- Z-String
%- Concept hierarchy
%- Differential privacy
%- Other classical techniques: Removal, Perturbation (Adding noise), Shifting (for example all timestamps to preserve interval lengths but hide real datetime)



Pseudonymisierung nicht für alle Felder sinnvoll, daher hier ergänzende Möglichkeiten und auch weitere Ansätze. 
Natürlich auch mehrere kombinierte Verfahren möglich bzw sinnvoll.



\section{Unterdrückung} % Suppression

Als ergänzende Maßnahme für Datenfelder, die für die Anomalieerkennung nicht benötigt werden, aber Rückschlüsse auf den Benutzer zulassen, kommt die Unterdrückung in Frage. Hierbei wird der Wert des Feldes schlicht entfernt oder durch eine Konstante ersetzt. \todo{Beispiel}

\section{Generalisierung}

Bei der Generalisierung wird der Feldinhalt durch Werte ersetzt, die das gleiche Konzept beschreiben, jedoch allgemeiner sind. Hierdurch entstehen Generalisierungshierarchien \todo{Beispiel, Diagramm?}, wobei die Stufe der höchsten Generalisierung in diesem Anwendungsfall gleichbedeutend mit der Unterdrückung ist, da jeder Wert durch den konstanten Wert der höchsten Generalisierungsstufe ersetzt wird. Ein Beispiel im Unternehmenskontext dieser Arbeit ist die Generalisierung eines Mitarbeiters zu seiner Arbeitsgruppe oder Abteilung sein -- eine Information, die für die Anomalieerkennung ausreichend sein könnte, wenn es beispielsweise um Zugriffe auf Ressourcen geht, die für bestimmte Abteilungen üblich, für andere jedoch ungewöhnlich sind. Dieses könnte zusätzlich zur Pseudonymisierung ausgeführt werden, um der Anomalieerkennung zusätzliche Daten zur Verfügung zu stellen, ohne die Identität eines Nutzers direkt offenzulegen.

\section{Verrauschen} % Hinzufügen von Rauschen

Diese Maßnahme verändert den Wert eines Datenfeldes, indem diesem Werte aus einer Wahrscheinlichkeitsverteilung hinzugefügt werden (statistisches Rauschen). Hierdurch lassen sich die Rückschlüsse auf einen Nutzer aus einem einzelnen Datensatz verringern, aber die Gesamtverteilung bleibt erhalten bzw. lässt sich leicht berechnen. Hierdurch lassen sich zumindest Abweichungen von Durchschnittswerten zur Anomalieerkennung nutzen. Es wird jedoch eine ausreichend große Datenmenge benötigt. Alternativ lassen sich zumindest Aussagen über den Bereich treffen, in dem ein Wert sich befinden muss. Dies könnte beispielsweise bei dem Verrauschen von Ereigniszeitstempeln sinnvoll sein, bei dem zwar nicht auf den konkreten Zeitpunkt geschlossen werden kann, aber zumindest Aussagen darüber getroffen werden können, ob das Ereignis in einem üblichen Intervall wie in den normalen Bürostunden auftrat.\\
Die Maßnahme ist jedoch nur für bestimmte Felder bzw. Datenarten sinnvoll einsatzbar. Gegenbeispiele sind unter anderem Freitextfelder wie Benutzernamen oder Felder für Aufzählungstypen wie Raumnummern. 

\section{Nutzung von Hashverfahren}

Neben der zufälligen Generierung von Pseudonymen, wie es in dieser Arbeit genutzt wird, ist auch die Nutzung von Hashwerten als Pseudonym für Daten denkbar. Dies würde die Verknüpfbarkeit von Logdaten ermöglichen, da für gleiche Daten der gleiche Hashwert berechnet wird. Durch den geschickten Einsatz von zusätzlichen zeitabhängig wechselnden Eingaben für das Hashverfahren (sogenannte \textit{Salts}) ließe sich auch die nötige Verknüpfbarkeit für die Anomalieerkennung gegenüber dem Schutz der Privatsphäre eines Benutzers abstimmen. Auf der anderen Seite wäre der Einsatz von Hashverfahren bei einem kleinen Wertebereich für Eingaben wie Benutzernamen wie in Abschnitt \ref{sec_state_se} beschrieben anfällig für Wörterbuchangriffe. Außerdem wären auch Rückschlüsse auf den Pseudonymhalter nicht ohne zusätzlichen Aufwand möglich. Für Datenfelder, bei denen nur die Verknüpfbarkeit, jedoch nicht der ursprüngliche Wert für die Anomalieerkennung entscheidend sind, könnten Hashverfahren sinnvoll sein. \todo{Beispiel}


\section{Vorgehen zur Integration}