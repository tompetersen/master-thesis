\section{Searchable Symmetric Encryption}

\label{sec_basisc_se}

%- Searchable Symmetric encryption vs Public Key Encryption With Keyword Search
%- Hier relevant SSE

\textit{Searchable Symmetric Encryption} (SSE) ist ein Konzept, das es ermöglicht, Daten in verschlüsselter Form auf einen Server auszulagern und trotzdem Suchanfragen auf den Daten ausführen zu können. Ein allgemeines SSE-Schema besteht aus vier effizient berechenbaren Algorithmen \cite{wang2016}:

\begin{itemize}
  \item \textbf{GenerateKey}\((k)\) generiert einen geheimen Schlüssel \(K\) anhand eines (verfahrensabhängigen) Sicherheitsparameters \(k\).
  \item \textbf{BuildIndex}\((K, D)\) erstellt einen Suchwort-Index \(I\) aus dem generierten Schlüssel \(K\) und einer Dokumentenmenge \(D\).
  \item \textbf{GenerateTrapdoor}\((K, w)\) erstellt für ein spezielles Suchwort \(w\) mithilfe des Schlüssels \(K\) das Trapdoor-Element \(T_w\) für die Suche nach \(w\).
  \item \textbf{Search}\((I, T_w)\) liefert eine Menge von Dokumenten basierend auf einem Suchwort-Index \(I\) und einem Trapdoor-Element \(T_w\).
\end{itemize}

Der Besitzer der Daten erstellt sich einen Schlüssel mithilfe von \textbf{GenerateKey} und generiert durch \textbf{BuildIndex} einen Suchwort-Index für seine Dokumente. Anschließend lädt er diese Dokumente in verschlüsselter Form zusammen mit dem Index auf den Server. 

Möchte der Besitzer nun alle Dokumente erhalten, auf die ein spezielles Suchwort zutrifft, so erstellt er für dieses Suchwort mithilfe von \textbf{GenerateTrapdoor} ein Trapdoor-Element und sendet dieses an den Server. 

Dort wird nun auf dem Suchwort-Index durch \textbf{Search} die Suche nach dem Trapdoor-Element ausgeführt, die eine Menge von verschlüsselten Dokumenten liefert, auf die das Suchwort zutrifft. Diese können zurück an den Besitzer gesendet werden, der sie lokal entschlüsseln kann. 