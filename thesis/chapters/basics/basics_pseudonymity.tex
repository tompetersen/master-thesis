\section{Pseudonymisierung}

\begin{itemize}
  \item \textbf{Grundlagen} Was ist Pseudonymisierung (auch Abgrenzung zu Anonymisierung)?
  \item \textbf{Eigenschaften} Welche Unterschiede gibt es innerhalb der Pseudonymisierung und wie können diese abgewogen werden (Abhängigkeit von Datenquelle, ...)?
\end{itemize}





\label{sec_basics_pseudonymity}

%- Pseudonymisierung als Möglichkeit der Verschleierung und Nicht-Verkettbarkeit.

\subsection{Grundlagen}

Pseudonymisierung beschreibt nach \cite{pfitzmann2001, pfitzmann2010} die Benutzung von Pseudonymen zur Identifizierung von Subjekten, wobei ein Pseudonym\footnote{
	ursprünglich aus dem Griechischen stammend: \textit{pseudonumon} - falsch benannt
} als Identifikator eines Subjekts ungleich seinem echten Bezeichner definiert wird. \todo{Pseudonymtypen}

\subsection{Eigenschaften der Pseudonymisierung}

Pseudonymität sagt dabei erst einmal lediglich etwas über die Verwendung eines Verfahrens aus, jedoch nichts über die daraus entstehenden Auswirkungen auf die Identifizierbarkeit eines Subjekts oder die Zurechenbarkeit bestimmter Aktionen. Hierfür spielen weitere Eigenschaften von Pseudonymen wie die folgenden eine Rolle:
\begin{itemize}
  \item garantierte Eindeutigkeit von Pseudonymen
  \item Möglichkeit von Pseudonymänderungen
  \item begrenzt häufige Verwendung von Pseudonymen 
  \item zeitlich begrenzte Verwendung von Pseudonymen
  \item Art der Pseudonymserstellung
\end{itemize}

Die Ausprägungen dieser Eigenschaften werden auch im Rahmen dieser Arbeit für das umzusetzende System zu bewerten sein.

% Nicht-Verkettbarkeit
% Aufdeckbarkeit
% Anonymität?
% 