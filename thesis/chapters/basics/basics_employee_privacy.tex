\section{Arbeitnehmerdatenschutz}

\label{sec_basics_employee_privacy}

\todo{Gesetzestexte als "`Quellen"` in Fußnoten?}

Der Begriff des Arbeitnehmerdatenschutzes\footnote{
  In manchen Veröffentlichungen wird der Arbeitnehmerdatenschutz auch als Mitarbeiterdatenschutz, Beschäftigtendatenschutz, Personaldatenschutz oder Betriebsdatenschutz bezeichnet.
}
beschreibt die Rechte von Arbeitnehmern im Beschäftigungsverhältnis im Bezug auf den Umgang mit personenbezogenen Daten. In diesem Abschnitt soll ein kompakter Überblick über aktuell geltende und in nächster Zeit in Kraft tretende gesetzliche Regelungen im Bezug hierauf gegeben werden, wobei der Fokus auf zum Thema der Arbeit passenden Regelungen liegt.

Zu Beginn soll kurz auf das Recht auf informationelle Selbstbestimmung eingegangen werden, das die Grundlage für alle folgenden Betrachtungen zum Arbeitnehmerdatenschutz bildet.

\subsection*{Das Recht auf informationelle Selbstbestimmung}

Im sogenannten Volkszählungsurteil aus dem Jahr 1983 wurde das Recht auf informationelle Selbstbestimmung als Grundrecht anerkannt \cite{TODO}. 
Es handelt sich um eine Ausprägung des allgemeinen Persönlichkeitsrechts\footnote{
  Das allgemeine Persönlichkeitsrecht beschreibt den Schutz der Persönlichkeit einer Person vor Eingriffen in ihren Lebens- und Freiheitsbereich.
} nach Artikel 2, Absatz 1 in Verbindung mit Artikel 1, Absatz 1 des Grundgesetzes. Es beschreibt das Recht des Einzelnen, selber über den Umgang mit seinen personenbezogenen Daten entscheiden zu können. 

%    \todo{Nötig?}
%    \begin{quotation}
%    Die Würde des Menschen ist unantastbar. Sie zu achten und zu schützen ist Verpflichtung aller staatlichen Gewalt.
%    
%    \textit{-- Artikel 1, Absatz 1, GG}
%    
%    Jeder hat das Recht auf die freie Entfaltung seiner Persönlichkeit, soweit er nicht die Rechte anderer verletzt und nicht gegen die verfassungsmäßige Ordnung oder das Sittengesetz verstößt. 
%        
%    \textit{-- Artikel 2, Absatz 1, GG} 
%    \end{quotation}
    
Mit dem vermehrten Aufkommen automatisierter Datenverarbeitung stellten die Richter des Bundesverfassungsgerichts damals die besondere Schutzbedürftigkeit der Selbstbestimmung des Einzelnen im Bezug auf die Offenbarung von Lebenssachverhalten heraus. Sie betonten die Notwendigkeit dieser Selbstbestimmung als Voraussetzung für eine freie Entfaltung der Persönlichkeit und auch für die Ausübung bestimmter Grundrechte wie der Versammlungsfreiheit. Damit sei das Recht auf informationelle Selbstbestimmung auch \glqq eine elementare Funktionsbedingung eines auf Handlungs- und Mitwirkungsfähigkeit seiner Bürger begründeten freiheitlichen demokratischen Gemeinwesens\grqq{} \cite{TODO} .
    
Einschränkungen dieses Rechts sind dem Urteil nach möglich, jedoch in Gesetzen festzuhalten. Hierbei müssen das Geheimhaltungsinteresse des Betroffenen und das öffentliche Informationsinteresse verarbeitender Stellen gegeneinander abgewogen werden.

Auch wenn sich das Urteil des Bundesverfassungsgerichts nur auf die Rechte des Betroffenen gegenüber staatlichen Akteuren bezieht, so bildet die Intention des Rechts auf informationelle Selbstbestimmung doch die Grundlage für das heutige Bundesdatenschutzgesetz und auch die Datenschutzgrundverordnung der EU, die auch für nicht-staatliche Akteure Gültigkeit besitzen.

Zusätzlich findet sich das Recht auf informationelle Selbstbestimmung auch in der Grundrechtecharta der EU: \glqq Jede Person hat das Recht auf Schutz der sie betreffenden personenbezogenen Daten.\grqq{}\todo{Quelle}\footnote{
  Artikel 8, Absatz 1
}

\subsection*{Datenschutz im Beschäftigungsverhältnis}

Eine besondere Situtation ergibt sich im Unternehmenskontext. Hier muss das Recht des Arbeitnehmers auf informationelle Selbstbestimmung gegenüber dem berechtigten Interesse des Arbeitgebers an der Aufklärung von Straftaten im Beschäftigungsverhältnis abgewogen werden. 

Im zur Zeit gültigen Bundesdatenschutzgesetz (BDSG) wird in § 4 die Erhebung, Verarbeitung und Nutzung personenbezogener Daten nur als zulässig angesehen, falls der Betroffene einwilligt oder ein Gesetz dieses erlaubt. Personenbezogene Daten werden in § 3 hierbei als \glqq Einzelangaben über [...] Verhältnisse einer bestimmten oder bestimmbaren natürlichen Person\grqq{} \todo{Quelle} definiert.

§ 32 beschreibt die Datenerhebung, -verarbeitung und -nutzung für Zwecke des Beschäftigungsverhältnisses:
\begin{quotation}
Personenbezogene Daten eines Beschäftigten dürfen für Zwecke des Beschäftigungsverhältnisses erhoben, verarbeitet oder genutzt werden, wenn dies für die Entscheidung über die Begründung eines Beschäftigungsverhältnisses oder nach Begründung des Beschäftigungsverhältnisses für dessen Durchführung oder Beendigung erforderlich ist.\\
Zur Aufdeckung von Straftaten dürfen personenbezogene Daten eines Beschäftigten nur dann erhoben, verarbeitet oder genutzt werden, wenn zu dokumentierende tatsächliche Anhaltspunkte den Verdacht begründen, dass der Betroffene im Beschäftigungsverhältnis eine Straftat begangen hat, die Erhebung, Verarbeitung oder Nutzung zur Aufdeckung erforderlich ist und das schutzwürdige Interesse des Beschäftigten an dem Ausschluss der Erhebung, Verarbeitung oder Nutzung nicht überwiegt, insbesondere Art und Ausmaß im Hinblick auf den Anlass nicht unverhältnismäßig sind.\footnote{
  § 32, Absatz 1, Bundesdatenschutzgesetz
}
\end{quotation}

Während sich der erste Satz auf den Umgang mit personenbezogenen Daten in einem normalen Beschäftigungsverhältnis befasst und bezogen auf das Thema dieser Arbeit beispielsweise den Rahmen für erforderliche Datenverarbeitung zur Aufdeckung von Vertragsbrüchen unterhalb der Straftatgrenze darstellt, behandelt der zweite Satz den Straftatfall. Hier sind insbesondere der notwendige Anfangsverdacht als Voraussetzung und die Verhältnismäßigkeit der Datennutzung zu beachten. 

Weiterhin insbesondere im Rahmen dieser Arbeit entscheidend ist die Ausrichtung des BDSG auf personenbezogene Daten, die wie bereits definiert einer bestimmbaren Person zugeordnet können werden müssen. Das in dieser Arbeit angestrebte System wird durch Pseudonymisierung und erst durch Kollaboration ermöglichte De-Pseudonymisierung den direkten Personenbezug verhindern und erst im durch mehrere Instanzen bestätigten Straftatverdacht ermöglichen\footnote{
  Der Autor maßt sich an dieser Stelle allerdings keine Beurteilung über die tatsächliche rechtliche Bewertung dieser Lösung an.
}.

2018 tritt die EU-Verordnung 2016/679, besser bekannt als Datenschutzgrundverordnung, in Kraft. In Deutschland wird das bestehende BDSG durch das Datenschutz-Anpassungs- und Umsetzungsgesetz grundlegend überarbeitet und an die Verordnung angepasst, um diese zu ergänzen. Hier finden sich in § 26 die Bestimmungen zur Datenverarbeitung für Zwecke des Beschäftigungsverhältnisses. Der zitierte Absatz aus dem BDSG ist dort in ähnlicher Form zu finden, wird also auch weiterhin seine Gültigkeit behalten. 

\tbc{Wie sollten Gesetzestexte zitiert werden?}

\todo{Beispiele für Überwachungsskandale}











\endinput

\begin{itemize}
  \item  Recht auf informationelle Selbstbestimmung als Ausprägung des Allgemeinen Persönlichkeitsrechts
  \item Besondere Rechtslage im Beschäftigungsverhältnis (aktuelle Rechtsprechung und Ausblick...)
  \item  (Antibeispiele: Lidl,  Bahn, Überwachungsaffäre der Deutschen Telekom)
  \item \textit{Eine heimliche Überwachung von Mitarbeitern ist im Regelfall unzulässig, wie das Bundesarbeitsgericht jüngst entschieden hat (Urteil vom 27. Juli 2017, 2 AZR 681/16). }
\end{itemize}


  
  \subsection*{Recht auf informationelle Selbstbestimmung}
    
    Das RaiS ist eine Ausprägung des allgemeinen Persönlichkeitsrechts (Schutz der Persönlichkeit einer Person vor Eingriffen in ihren Lebens- und Freiheitsbereich) nach Artikel 2, Absatz 1 in Verbindung mit Artikel 1, Absatz 1 des Grundgesetzes.
    
    \begin{quotation}
    Die Würde des Menschen ist unantastbar. Sie zu achten und zu schützen ist Verpflichtung aller staatlichen Gewalt.
    
    \textit{-- Artikel 1, Absatz 1, GG}
    
    Jeder hat das Recht auf die freie Entfaltung seiner Persönlichkeit, soweit er nicht die Rechte anderer verletzt und nicht gegen die verfassungsmäßige Ordnung oder das Sittengesetz verstößt. 
        
    \textit{-- Artikel 2, Absatz 1, GG} 
    \end{quotation}
    
    Es wurde im Volkszählungsurteil als Grundrecht anerkannt.
    
    \begin{quotation}
    TBD siehe pdf markiert
    
    \textit{-- Volkszählungsurteil}
    \end{quotation}
    
    Dieses Urteil enthält auch erforderliche Grundsätze bei der Datenverarbeitung wie Datensparsamkeit, Zweckgebundenheit, ... (richtet sich jedoch nur an staatliche Eingriffe)
    
    Einschränkungen sind möglich, jedoch in Gesetzen festzuhalten (Abwägung zwischen Geheimhaltungsinteresse des Betroffenen und dem öffentlichen Informationsinteresse verarbeitender Stellen).
    
    Das RiaS bildet die Grundlage für Gesetze wie das BDSG, die LDSG (diese jedoch nur für öffentlich-rechtliche Einrichtungen relevant) oder DSGVO der EU.
    
  \subsection*{Art.8, EU-Grundrechtecharta}
      
      \textbf{Schutz personenbezogener Daten}
      
      (1) Jede Person hat das Recht auf Schutz der sie betreffenden personenbezogenen Daten. 
      
      (2) Diese Daten dürfen nur nach Treu und Glauben für festgelegte Zwecke und mit Einwilligung 
      der betroffenen Person oder auf einer sonstigen gesetzlich geregelten legitimen Grundlage verarbeitet werden. Jede Person hat das Recht, Auskunft über die sie betreffenden erhobenen Daten zu erhalten und die Berichtigung der Daten zu erwirken. 
      
      (3) Die Einhaltung dieser Vorschriften wird von einer unabhängigen Stelle überwacht. 
  
  \subsection*{§32, Bundesdatenschutzgesetz}
  
  Ursprünglich:
   
  https://dejure.org/gesetze/BDSG/32.html
  
  \begin{quotation}
  
  \textbf{Datenerhebung, -verarbeitung und -nutzung für Zwecke des Beschäftigungsverhältnisses}
  
  (1) Personenbezogene Daten eines Beschäftigten dürfen für Zwecke des Beschäftigungsverhältnisses erhoben, verarbeitet oder genutzt werden, wenn dies für die Entscheidung über die Begründung eines Beschäftigungsverhältnisses oder nach Begründung des Beschäftigungsverhältnisses für dessen Durchführung oder Beendigung erforderlich ist. \\
  \textbf{Zur Aufdeckung von Straftaten dürfen personenbezogene Daten eines Beschäftigten nur dann erhoben, verarbeitet oder genutzt werden, wenn zu dokumentierende tatsächliche Anhaltspunkte den Verdacht begründen, dass der Betroffene im Beschäftigungsverhältnis eine Straftat begangen hat, die Erhebung, Verarbeitung oder Nutzung zur Aufdeckung erforderlich ist und das schutzwürdige Interesse des Beschäftigten an dem Ausschluss der Erhebung, Verarbeitung oder Nutzung nicht überwiegt, insbesondere Art und Ausmaß im Hinblick auf den Anlass nicht unverhältnismäßig sind.}
  
  (2) Absatz 1 ist auch anzuwenden, wenn personenbezogene Daten erhoben, verarbeitet oder genutzt werden, ohne dass sie automatisiert verarbeitet oder in oder aus einer nicht automatisierten Datei verarbeitet, genutzt oder für die Verarbeitung oder Nutzung in einer solchen Datei erhoben werden.
  
  (3) Die Beteiligungsrechte der Interessenvertretungen der Beschäftigten bleiben unberührt.
  
  \end{quotation}
  
  Mögliche Erweiterung durch Entwurf:

  Deutscher Bundestag: Gesetzentwurf der Bundesregierung: Entwurf eines Gesetzes zur Regelung des Beschäftigtendatenschutzes, BT-Drs. 17/4230 vom 15. Dezember 2010
  
  http://dip21.bundestag.de/dip21/btd/17/042/1704230.pdf
  
  http://www.arbeitnehmerdatenschutz.de/
  
  Bisher scheinbar keine Abstimmung, da auf die EU DSGVO gewartet wurde.
  
  \subsection*{§26, BDSG(neu) - Datenschutz-Anpassungs- und Umsetzungsgesetz} 

    Tritt Mai 2018 in Kraft.
    
    \begin{quotation}
      \textbf{Datenverarbeitung für Zwecke des Beschäftigungsverhältnisses}
    
        (1) Personenbezogene Daten von Beschäftigten dürfen für Zwecke des Beschäftigungsverhältnisses verarbeitet werden, wenn dies für die Entscheidung über die Begründung eines Beschäftigungsverhältnisses oder nach Begründung des Beschäftigungsverhältnisses für dessen Durchführung oder Beendigung oder zur Ausübung oder Erfüllung der sich aus einem Gesetz oder einem Tarifvertrag, einer Betriebs- oder Dienstvereinbarung (Kollektivvereinbarung) ergebenden Rechte und Pflichten der Interessenvertretung der Beschäftigten erforderlich ist. Zur Aufdeckung von Straftaten dürfen personenbezogene Daten von Beschäftigten nur dann verarbeitet werden, wenn zu dokumentierende tatsächliche Anhaltspunkte den Verdacht begründen, dass die betroffene Person im Beschäftigungsverhältnis eine Straftat begangen hat, die Verarbeitung zur Aufdeckung erforderlich ist und das schutzwürdige Interesse der oder des Beschäftigten an dem Ausschluss der Verarbeitung nicht überwiegt, insbesondere Art und Ausmaß im Hinblick auf den Anlass nicht unverhältnismäßig sind.
        
        (2) Erfolgt die Verarbeitung personenbezogener Daten von Beschäftigten auf der Grundlage einer Einwilligung, so sind für die Beurteilung der Freiwilligkeit der Einwilligung insbesondere die im Beschäftigungsverhältnis bestehende Abhängigkeit der beschäftigten Person sowie die Umstände, unter denen die Einwilligung erteilt worden ist, zu berücksichtigen. Freiwilligkeit kann insbesondere vorliegen, wenn für die beschäftigte Person ein rechtlicher oder wirtschaftlicher Vorteil erreicht wird oder Arbeitgeber und beschäftigte Person gleichgelagerte Interessen verfolgen. Die Einwilligung bedarf der Schriftform, soweit nicht wegen besonderer Umstände eine andere Form angemessen ist. Der Arbeitgeber hat die beschäftigte Person über den Zweck der Datenverarbeitung und über ihr Widerrufsrecht nach Artikel 7 Absatz 3 der Verordnung (EU) 2016/679 in Textform aufzuklären.
        
        (3) Abweichend von Artikel 9 Absatz 1 der Verordnung (EU) 2016/679 ist die Verarbeitung besonderer Kategorien personenbezogener Daten im Sinne des Artikels 9 Absatz 1 der Verordnung (EU) 2016/679 für Zwecke des Beschäftigungsverhältnisses zulässig, wenn sie zur Ausübung von Rechten oder zur Erfüllung rechtlicher Pflichten aus dem Arbeitsrecht, dem Recht der sozialen Sicherheit und des Sozialschutzes erforderlich ist und kein Grund zu der Annahme besteht, dass das schutzwürdige Interesse der betroffenen Person an dem Ausschluss der Verarbeitung überwiegt. Absatz 2 gilt auch für die Einwilligung in die Verarbeitung besonderer Kategorien personenbezogener Daten; die Einwilligung muss sich dabei ausdrücklich auf diese Daten beziehen. § 22 Absatz 2 gilt entsprechend.
        
        (4) Die Verarbeitung personenbezogener Daten, einschließlich besonderer Kategorien personenbezogener Daten von Beschäftigten für Zwecke des Beschäftigungsverhältnisses, ist auf der Grundlage von Kollektivvereinbarungen zulässig. Dabei haben die Verhandlungspartner Artikel 88 Absatz 2 der Verordnung (EU) 2016/679 zu beachten.
        
        (5) Der Verantwortliche muss geeignete Maßnahmen ergreifen, um sicherzustellen, dass insbesondere die in Artikel 5 der Verordnung (EU) 2016/679 dargelegten Grundsätze für die Verarbeitung personenbezogener Daten eingehalten werden.
        
        (6) Die Beteiligungsrechte der Interessenvertretungen der Beschäftigten bleiben unberührt.
        
        (7) Die Absätze 1 bis 6 sind auch anzuwenden, wenn personenbezogene Daten, einschließlich besonderer Kategorien personenbezogener Daten, von Beschäftigten verarbeitet werden, ohne dass sie in einem Dateisystem gespeichert sind oder gespeichert werden sollen.
        
        (8) Beschäftigte im Sinne dieses Gesetzes sind:
        \begin{enumerate}
          \item Arbeitnehmerinnen und Arbeitnehmer, einschließlich der Leiharbeitnehmerinnen und Leiharbeitnehmer im Verhältnis zum Entleiher,
          \item zu ihrer Berufsbildung Beschäftigte,
          \item Teilnehmerinnen und Teilnehmer an Leistungen zur Teilhabe am Arbeitsleben sowie an Abklärungen der beruflichen Eignung oder Arbeitserprobung (Rehabilitandinnen und Rehabilitanden),
          \item in anerkannten Werkstätten für behinderte Menschen Beschäftigte,
          \item Freiwillige, die einen Dienst nach dem Jugendfreiwilligendienstegesetz oder dem Bundesfreiwilligendienstgesetz leisten,
          \item Personen, die wegen ihrer wirtschaftlichen Unselbständigkeit als arbeitnehmerähnliche Personen anzusehen sind; zu diesen gehören auch die in Heimarbeit Beschäftigten und die ihnen Gleichgestellten,
          \item Beamtinnen und Beamte des Bundes, Richterinnen und Richter des Bundes, Soldatinnen und Soldaten sowie Zivildienstleistende. 
        \end{enumerate}
        Bewerberinnen und Bewerber für ein Beschäftigungsverhältnis sowie Personen, deren Beschäftigungsverhältnis beendet ist, gelten als Beschäftigte.
    \end{quotation}
  
    
  \subsection*{EU Verordnung 2016/679 (Datenschutzgrundverordnung)}
  
    Verordnung (EU) 2016/679 des Europäischen Parlaments und des Rates vom 27. April 2016 zum Schutz natürlicher Personen bei der Verarbeitung personenbezogener Daten, zum freien Datenverkehr und zur \textbf{Aufhebung der Richtlinie 95/46/EG} (Datenschutz-Grundverordnung)
    
    Ersetzt: \textit{Richtlinie 95/46/EG des Europäischen Parlaments und des Rates vom 24. Oktober 1995 zum Schutz natürlicher Personen bei der Verarbeitung personenbezogener Daten und zum freien Datenverkehr}
    
    https://www.datenschutzbeauftragter-info.de/eu-datenschutz-grundverordnung-und-beschaeftigtendatenschutz/
      
    \begin{quotation}
      \textbf{Art. 88, Datenverarbeitung im Beschäftigungskontext}
    
      (1) Die Mitgliedstaaten können durch Rechtsvorschriften oder durch Kollektivvereinbarungen spezifischere Vorschriften zur Gewährleistung des Schutzes der Rechte und Freiheiten hinsichtlich der Verarbeitung personenbezogener Beschäftigtendaten im Beschäftigungskontext, insbesondere für Zwecke der Einstellung, der Erfüllung des Arbeitsvertrags einschließlich der Erfüllung von durch Rechtsvorschriften oder durch Kollektivvereinbarungen festgelegten Pflichten, des Managements, der Planung und der Organisation der Arbeit, der Gleichheit und Diversität am Arbeitsplatz, der Gesundheit und Sicherheit am Arbeitsplatz, des Schutzes des Eigentums der Arbeitgeber oder der Kunden sowie für Zwecke der Inanspruchnahme der mit der Beschäftigung zusammenhängenden individuellen oder kollektiven Rechte und Leistungen und für Zwecke der Beendigung des Beschäftigungsverhältnisses vorsehen.
      
      (2) Diese Vorschriften umfassen angemessene und besondere Maßnahmen zur Wahrung der menschlichen Würde, der berechtigten Interessen und der Grundrechte der betroffenen Person, insbesondere im Hinblick auf die Transparenz der Verarbeitung, die Übermittlung personenbezogener Daten innerhalb einer Unternehmensgruppe oder einer Gruppe von Unternehmen, die eine gemeinsame Wirtschaftstätigkeit ausüben, und die Überwachungssysteme am Arbeitsplatz.
      
      (3) Jeder Mitgliedstaat teilt der Kommission bis zum 25. Mai 2018 die Rechtsvorschriften, die er aufgrund von Absatz 1 erlässt, sowie unverzüglich alle späteren Änderungen dieser Vorschriften mit.
    \end{quotation}    