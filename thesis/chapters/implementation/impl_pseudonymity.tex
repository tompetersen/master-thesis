\section{Umsetzung der Pseudonymisierung} % und Searchable Encryption

\label{sec_impl_pseudonymity}

% Pseudonymgenerierung

% Parameter: Zeitabhängig, Häufigkeitsabhängig, USE_PFP
% Wie erfolgt die Konfiguration? Wo werden Werte gesetzt/gespeichert?

% MAC Generierung (bzw. Empfang als Search_token) beschreiben

% Zweiter MAC?
% Auch auf Vertrauensmodell im Zusammenhang mit PFP eingehen: Was erfährt die DB, was lässt sich verketten, was würde bei Aufdecken eines Pseudonyms geschehen? Was passiert bei unerlaubtem Zugriff auf gespeicherte Daten?

\todo{Um Implementation erweitern (welche Parameter in welchen Systemteilen, Setup, ...)}

Für die Pseudonymisierung wurde ein Plugin für den Proxy entwickelt, wie in Abschnitt \ref{sec_impl_integration_into_ossim} beschrieben. Bei jedem eintreffenden Logdatum kann abhängig von dem Datenformat für ein entsprechendes Datenfeld ein Pseudonym als Ersatz für das echte Datum gesetzt werden. Dazu stellt der Pseudonym-Service eine Schnittstelle bereit, über die für ein Datum ein Pseudonym erhalten werden kann. Durch diese Trennung wird eine höhere Sicherheit der Zuordnung zwischen Datum und Pseudonym erreicht: In dem Proxy kommen die Logdaten in unveränderter Form an und werden verändert weitergesendet, daher ist die Zuordnung hier implizit bekannt und muss in Kauf genommen werden. Die Speicherung dieser Zuordnung erfolgt jedoch nur in dem Pseudonym-Service. Durch die Verschlüsselung und die in den folgenden Abschnitten beschriebene MAC-abhängige Pseudonymgenerierung erfährt der Pseudonym-Service nichts über das Datum, das durch das Pseudonym beschrieben wird. So führt unberechtigter Zugriff auf die Datenbank des Pseudonym-Service nicht zu mehr Informationen über das Datum, das ein Pseudonym beschreibt.

\subsection{Setup-Phase}

Vor Verwendung der Pseudonymisierung müssen die Parameter zur Pseudonymgenerierung (vgl. Abschnitt \ref{sec_state_pseudonymity}) dem System bekannt gemacht werden. Diese Parameter können in dem Pseudonym-Service mittels einer Konfigurationsoberfläche durch einen berechtigten Benutzer gesetzt werden. Die für den Proxy relevanten Parameter können anschließend über eine Schnittstelle abgefragt werden. Vorerst handelt es sich hierbei lediglich um das maximale Zeitintervall, in dem ein Pseudonym genutzt werden darf.

\subsection{Proxy}

Beim Start des Proxies wird zuerst der gerade beschriebene Parameter erhalten. Anschließend können eintreffende Logdaten verarbeitet werden. Das eintreffende Datum wird verschlüsselt (siehe dazu Abschnitt \ref{sec_impl_threshold}) und anschließend zusammen mit einem über das Datum generierten MAC, der wie in Abschnitt \ref{sec_state_se} beschrieben für die Überprüfung auf bereits bestehende Pseudonyme genutzt wird, an den Pseudonym-Service gesendet. Das ursprüngliche Datum wird nun durch das gelieferte Pseudonym (siehe nächsten Abschnitt) ersetzt und das so veränderte Logdatum wird an das SIEM-System weitergeleitet.

Der Schlüssel, der für die Generierung des MACs verwendet wird, wird abhängig von dem erhaltenen Parameter nach einer bestimmten Zeitspanne neu generiert. Durch diesen Schlüsselwechsel wird wie beschrieben erreicht, das für gleiche Daten, für die der MAC mit einem neuen Schlüssel erstellt wird, auch neue Pseudonyme erhalten werden. 
Da der Schlüsselwechsel nicht Pseudonym-abhängig geschieht, ist die Zeitspanne global für alle Pseudonyme gültig und somit als maximale Zeitspanne zu verstehen. Dies kann für die Anomalieerkennung evtl. Probleme bereiten, wenn nicht genügend lange Überwachungsdaten verkettet werden können. Auf der anderen Seite würde eine Verweildauer für einzelne Pseudonyme ein Erfassen des Erstellungszeitpunkts in der Datenbank erfordern, was wie in Abschnitt \ref{sec_state_pseudonymity} beschrieben Rückschlüsse auf das ursprüngliche Datum des Pseudonyms liefern könnte. Daher wurde sich gegen diesen Ansatz entschieden.

\subsection{Service}

Auf der Service-Seite wird anhand des empfangenen MACs durch Vergleich mit in der Datenbank vorliegenden MACs überprüft, ob bereits ein Pseudonym für das eintreffende Datum vergeben wurde, das noch nicht zu häufig verwendet wurde. Hierzu wird der in der Konfiguration gesetzte Parameter zur maximalen Nutzungshäufigkeit von Pseudonymen verwendet. Liegt kein solches Pseudonym vor, so wird ein noch nicht verwendetes, zufälliges Pseudonym erstellt und zusammen mit dem MAC und dem verschlüsselten Datum in der Datenbank gespeichert. Anderenfalls wird das bereits vergebene Pseudonym zurückgeliefert. 

\subsection{Perfect Forward Privacy}

Im Zusammenspiel dieser Parameter kann jedoch noch ein Problem entstehen. Für neu vergebene Pseudonyme, die innerhalb eines Zeitabschnitts durch Überschreiten der maximalen Nutzungsanzahl entstanden sind, liegen in der Datenbank Einträge mit gleichem MAC vor. Auf diese Weise wird die Verknüpfung verschiedener Pseudonyme ermöglicht, wenn jemand (berechtigt oder unberechtigt) Zugriff auf die Daten erhält. Das Aufdecken eines Pseudonyms deckt auch alle anderen in diesem Zeitintervall erstellten Pseudonyme implizit auf, was dem in Abschnitt \ref{sec_state_pseudonymity} beschriebenen Prinzip der \textit{Perfect Forward Privacy} widerspricht. 

Dieses Problem könnte durch eine Hashwert-Berechnung für den eintreffenden MAC auf der Service-Seite verhindert werden, die einen Pseudonym-abhängigen Zufallswert (eine Art Salt) einbezieht. Hierdurch enthalten Datenbankeinträge, die innerhalb eines Zeitintervalls zu dem gleichen Datum gehören und daher den gleichen MAC besitzen, durch den einfließenden Zufallswert unterschiedliche Hashwerte. Durch die Einweg-Eigenschaft der Hashfunktion wäre die Verkettbarkeit verschiedener Pseudonyme verhindert. 
Jedoch erfordert dieser Ansatz eine Hash-Berechnung pro Datenbankeintrag für jede Anfrage und ist daher aus Performance-Sicht kritisch zu betrachten. Aus diesem Grund wird diese Möglichkeit vorerst nicht implementiert. Das bestehende Problem ist jedoch für ein konkretes Anwendungsszenario und bei der Wahl der Parameter -- insbesondere des Zeitintervalls -- zu beachten.
