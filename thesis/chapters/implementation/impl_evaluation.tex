\section{Evaluation}

\label{sec_impl_evaluation}

\subsection{Erfüllung der Anforderungen}

\begin{itemize}
  \item Geeignete Stelle zum Eingriff in den Datenfluss zwischen Logdatenquelle und SIEM-System
  \item Parameterabhängige Generierung eindeutiger, aber in gewissem Rahmen verknüpfbarer Pseudonyme
  \item Sicherer, verteilter Einsatz eines anpassbaren kryptographischen Schwellwertschemas -- vorzugsweise mit verteilter Schlüsselgenerierung
  \item Geeignete Benutzerinteraktion mit dem System an notwendigen Stellen
  \item Erweiterbarkeit um unbekannte Datenquellen
  \item Erweiterbarkeit um weitere Datenschutztechniken
\end{itemize}

...

\subsection{Angriffsmöglichkeiten}

- Zentrale Schlüsselgenerierung (Verweis auf state-distributed)

- Krypto nicht von Kryptographen überprüft (Sidechannel, ...)

- Mögliche Schwächen (sowohl Architektur als auch Krypto, Bezug zum Angreifermodell)

- Scheme leakt Nachrichtenlänge als Vielfaches der Blocklänge -> Paddingscheme?

- Reidentification-Problem (siehe auch Lundin 4.3)

\subsection{Performance}

- Zeitmessungen für einzelne Systemfunktionen

- Performance bzw. Auswirkungen der Nutzung (theoretische Rechenlast, Zeit-/Lastmessung, zusätzlicher Speicherbedarf, ...)