\section{Anforderungen}

\label{sec_impl_requirements}

Der umgesetzte Prototyp soll es ermöglichen, aufbauend auf dem bestehenden Open-Source-SIEM-System OSSIM Logdaten mittels Pseudonymisierung und Schwellwertschemata so zu verändern, dass diese erst durch Kollaboration einer bestimmten Anzahl an Teilnehmern wieder aufgedeckt werden können. Neben dieser primären Anforderung sollte er noch weitere wie die Erweiterbarkeit um weitere Dateschutztechniken erfüllen. Alle diese Anforderungen sollen im folgenden Abschnitt näher erläutert werden. Zuerst wird jedoch ein Angreifermodell für die prototypische Umsetzung aufgestellt.

\subsection*{Zugrundeliegendes Angreifermodell}

% Das Angreifermodell definiert die	maximal	berücksichtigte	Stärke eines Angreifers, gegen den ein	Schutzmechanismus	gerade noch wirkt.	
%Es beschreibt:
%–  Rollen des Angreifers (Außenstehender, Benutzer, Betreiber, Wartungsdienst, Produzent, Entwerfer …), auch kombiniert 
%–  Verbreitung des Angreifers (Stellen im System, an denen der Angreifer Informationen gewinnen oder Systemzustände verändern kann) 
%–  Verhalten des Angreifers 
%  •  passiv / aktiv,  beobachtend / verändernd 
%–  Rechenkapazität des Angreifers 
%  •  unbeschränkt: informationstheoretisch 
%  •  beschränkt: komplexitätstheoretisch 

Im Kontext dieser Arbeit, die sich mit der datenschutzfreundlichen Speicherung von Überwachungsdaten beschäftigt, muss zuerst folgende Vorüberlegung getroffen werden: Logdaten erreichen das verwendete SIEM-System abhängig von den verwendeten Protokollen im allgemeinen nicht-pseudonymisiert und oftmals weder verschlüsselt noch mit geschützer Integrität über das Netzwerk. \todo{Angriffsmöglichkeiten erläutern: Mitschneiden aller Daten und damit  Aufdecken von Pseudonymen, ...}
Diese Angriffsmöglichkeiten zu verhindern, ist ausdrücklich kein Ziel dieser Arbeit. Daher bezieht sich das Angreifermodell auf die Bearbeitung und Speicherung von Logdaten, erst sobald sie das zu entwickelnde System erreichen, jedoch nicht vorher.

Das Ziel des Systems bezogen auf seine Sicherheit lässt sich folgendermaßen definieren: Das Pseudonym eines Nutzers erlaubt (ohne Anwendung von Hintergrundwissen) keinen Rückschluss auf die Identität eines Nutzers. Erst die Kollaboration berechtigter Akteure ermöglicht das Aufdecken eines Pseudonyms.\todo{? außerdem erweitern um aktive Angreifer?}

Ein Angreifermodell beschreibt die maximale Stärke eines Angreifers im Bezug auf verschiedene Faktoren, gegen die ein System abgesichert ist. Enthalten sind die Rolle eines Angreifers, seine Verbreitung im System, aktives oder passives Verhalten und die Rechenkapazität, die der Angreifer zum Überwinden der eingesetzten Schutzmaßnahmen aufbringen kann. Es bildet die Basis für alle Folgeüberlegungen im Bezug auf die Sicherheit des zu entwickelnden Systems.

Für das Angreifermodell werden folgende Annahmen getroffen: Bei dem Angreifer kann es sich um einen Außenstehenden, um einen Berechtigten mit Zugriff auf das SIEM-System oder sogar um einen Administrator mit physischem Zugriff auf Rechner im Netz handeln. Wie bereits beschrieben, wird die ungesicherte Übertragung nicht-pseudonymisierter Daten zu dem System nicht betrachtet. Insofern wird von einem Angreifer ausgegangen, der erst Zugriff auf das zu entwickelnde System oder das SIEM-System beseitzt oder erlangt. Das zu entwickelnde System soll in der Lage sein, sich auch gegen aktive Angreifer zur Wehr zu setzen. \todo{Denial of Service etc?}
Bezogen auf die verfügbare Rechenleistung des Angreifers sollen verbreitete und nach heutigem Wissensstand für sicher befundene kryptographische Algorithmen als nicht mit vertretbarem Aufwand zu brechen angesehen werden. Es handelt sich daher um die Annahme von komplexitätstheoretischer Sicherheit.


\subsection*{Anforderungen an die Integration in OSSIM}

\label{sec_requirements_ossim_integration}

Um zu beurteilen, an welcher Stelle in den Datenfluss der Logdaten in OSSIM (vergleiche Abschnitt \ref{sec_state_siem}) eingegriffen wird, um die Daten zu verändern, müssen Vor- bzw. Nachteile der verschiedenen Möglichkeiten gegeneinander abgewogen werden. Folgende Eigenschaften einer Möglichkeit sollten betrachtet werden:

\begin{itemize}

  \item \textbf{Abhängigkeit von der OSSIM-Konfiguration: } Ist die Lösung sowohl in der verteilten als auch in der All-in-One-Installation von OSSIM umsetzbar? Eine Lösung, die unabhängig von der gewählten Installationsmethode funktioniert, ist zu bevorzugen, da sie universell einsetzbar ist.
  
  \item \textbf{Veränderung des SIEM-Systems: } Muss OSSIM für die Umsetzung der Lösung verändert werden? Dies wäre im Hinblick auf zukünftige Updates, die OSSIM durch den Entwickler erfährt, nicht wünschenswert, da jedes dieser Updates dafür sorgen könnte, dass die umgesetzte Lösung angepasst werden muss. Hierbei können sowohl direkte Änderungen an dem OSSIM-Quellcode als auch Eingriff in die interne Kommunikation betrachtet werden.  
  
  \item \textbf{Nicht-pseudonymisierte Daten im SIEM-System: } Um das Ziel der Arbeit - die Pseudonymisierung, die nur durch Kollaboration aufgedeckt werden kann - zu erreichen, muss sichergestellt sein, dass Logdaten nirgendwo in nicht pseudonymisierter Form vorliegen. Da insbesondere das zukünftige Verhalten von OSSIM nicht beeinflusst werden kann, wäre es wünschenswert, dass die Logdaten bereits in pseudonymisierter Form das SIEM-System erreichen.\\
  Die Relevanz dieser Eigenschaft lässt sich bereits an einem einfachen Beispiel erkennen: Wird das Syslog-Protokoll genutzt, um Logdaten in OSSIM aufzunehmen, so werden die Einträge im Sensor erst in eine Logdatei gespeichert und von dort aus geparst, normalisiert und in der Datenbank gespeichert. Das Datum verbleibt in der Logdatei. Kommen die Daten in nicht-pseudonymisierter Form in dem OSSIM-Sensor an, so muss sichergestellt werden, dass verarbeitete Einträge gelöscht/verändert werden.
  
  \item \textbf{Mehrfaches Parsen von Logdaten: } Durch den OSSIM-Agenten werden die Logdaten - wie in Abschnitt \ref{sec_state_siem} beschrieben - geparst und normalisiert. Aus Performancegründen ist eine Lösung zu bevorzugen, die dieses Parsen nicht mehrfach voraussetzt.
  
%  \item \textbf{Manipulierbarkeit auf den Übertragungswegen(?): }

\end{itemize}

\subsection*{Anforderungen an die Pseudonymisierung}

%- Lang genug für geringe Kollisionswsk.
%- Eindeutig
%- Durchsuchbar (mim Hinblick auf threshold)
%- Anwendungsfallabhängige Parameter für Nutzzeit, ... (Rückblick auf Kapitel 3)

Die Pseudonymisierung muss es ermöglichen nach Aufdecken eines Eintrags wieder auf den ursprünglichen Dateninhalt schließen zu können. Daher müssen die Pseudonyme für die Zeit ihrer Speicherung eindeutig sein, d.h. es darf zu keiner gleichzeitigen Mehrfachverwendung von Pseudonymen kommen. 

Weiterhin muss es eine Möglichkeit beim Pseudonymisieren von Logeinträgen geben, zu überprüfen, ob für ein Datum bereits ein Pseudonym vergeben wurde. So kann sichergestellt werden, dass in einem bestimmten Zeitraum Logeinträge zu einer Person mit dem gleichen Pseudonym versehen werden, um über die Verknüpfung von Einträgen die gewünschte Erkennung von Insider-Angriffen erreichen zu können.

Außerdem muss es eine Möglichkeit geben, die Parameter der Pseudonymisierung wie den Zeitraum ihrer Verwendung (vergleiche Abschnitt \ref{sec_state_pseudonymity}) konfigurierbar zu machen.

\subsection*{Anforderungen im Bezug auf den Einsatz eines kryptographischen Schwellwertschemas}

%- Verteiltes Modell 
%- Kommunikation
%- Schlüsselmanagement
%- ...

\todo{Anpassen nach Kapitel 3 - Threshold-Abschnitt?}

Der Einsatz eines kryptographischen Schwellwertschemas setzt eine verteilte Anwendung voraus, die den Zugriff für die Pseudonymisierungskomponente sowie für die bei der Entschlüsselung eines Eintrags beteiligten Akteure bereitstellt. Die für das Schwellwertschema nötigen Parameter \(t\) und \(n\) sowie die beteiligten Akteure müssen anpassbar bzw auswählbar sein.

In der Phase der Schlüsselgenerierung muss das System die Kommunikation und Koordination aller Beteiligten unterstützen. Die hier erstellten Schlüssel und \textit{Shares} müssen an geeigneten Stelle sicher gespeichert und abrufbar sein.

Der für die Verschlüsselung erforderliche öffentliche Schlüssel muss so vorliegen, dass er bei der Verschlüsselung eines Pseudonym-Datensatzes genutzt werden kann.

Bei der Entschlüsselung eines Eintrags, also der Aufdeckung eines Pseudonyms, muss das System wiederum die beteiligten Aktuere koordinieren und anschließend die Rolle des \textit{Combiners} übernehmen, so dass anschließend der entschlüsselte Datensatz zentral vorliegt.

\subsection*{Benutzerinteraktion}

%- Proxy: Konfiguration der Plugins

%- Pseudo-App: Statusanzeige angemeldeter Benutzer(Admin), Initialisierung der Schlüsselgenerierung nach Nutzerauswahl(Admin), Anlegen von Aufdeckanfragen, Statusanzeige von Aufdeckanfragen, Systemstatus

%- Einzelne Teilnehmer sollten Client-Anwendungen besitzen, um auf Anfragen reagieren zu können (Generation eigener Schlüssel, gemeinsame Schlüsselgenerierung, ... ) Konsolenanwendung? 

Die zu entwickelnde verteilte Anwendung wird an verschiedenen Stellen Benutzerinteraktion erfordern.

Das Konfigurieren des Systems zur Integration verschiedener Datenquellen und deren Beschreibung muss einem berechtigten Nutzer zugänglich gemacht werden. Ebenso sollte es im Hinblick auf die in der Aufgabenstellung geforderte Erweiterbarkeit im Bezug auf weitere Datenschutztechniken relativ leicht sein, diese Techniken im System nutzen zu können. 

Für pseudonymisierte Datensätze muss es berechtigten Benutzern ermöglicht werden, Anfragen zur Aufdeckung eines Pseudonyms zu stellen und sich über ihren Status informiert zu halten.

Für die Benutzung eines kryptographischen Schwellwertschemas sollte es einem Administrator des Systems ermöglicht werden, grundlegende Parameter des Systems wie die Schwellwertparameter und die beteiligten Nutzer auszuwählen sowie die Initialisierung des Schemas anzustoßen. \\
Die am Schwellwertschema beteiligten Nutzer müssen die Möglichkeit erhalten, eine Übersicht über sie betreffende Anfragen zur Aufdeckung eines Pseudonym-Datensatzes zu bekommen sowie einzelne Anfragen abzulehnen oder sich an m Prozess des Aufdeckens mithilfe des Schwellwertschemas zu beteiligen. 

\todo{Performance?}

\subsection*{Erweiterbarkeit um neue Datenquellen}

Das umzusetzende System sollte es ermöglichen, Daten aus verschiedenen Quellen und (abhängig vom gewählten Eingriffspunkt in OSSIM auch) in verschiedenen Formaten entgegenzunehmen und mithilfe der umgesetzten Datenschutztechniken verändern zu können. Dabei muss das Format der Logdaten grundsätzlich beibehalten werden, um die Behandlung der Daten in OSSIM weiterhin zu ermöglichen.

\subsection*{Erweiterbarkeit um neue Datenschutztechniken}

Neben der im Fokus dieser Arbeit stehenden Pseudonymisierung und dem Einsatz von kryptographischen Schwellwertschemata zum Schutz der Logdaten gibt es weitere Datenschutztechniken, die für den Anwendungsfall genutzt werden könnten (siehe Kapitel \ref{cha_alternatives}). Der umgesetzte Prototyp sollte leicht um diese Techniken erweiterbar sein, d.h. so gestaltet sein, dass andere Techniken ohne große Änderungen am System integriert und auf eingehende Logdaten angewendet werden können.

