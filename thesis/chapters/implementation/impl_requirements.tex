\section{Anforderungen}

\label{sec_impl_requirements}

Der umgesetzte Prototyp soll es ermöglichen, aufbauend auf dem bestehenden Open-Source-SIEM-System OSSIM Logdaten mittels Pseudonymisierung und Schwellwertschemata so zu verändern, dass diese erst durch Kollaboration einer bestimmten Anzahl an Teilnehmern wieder aufgedeckt werden können.






\subsection*{Zugrundeliegendes Angreifermodell}

% Das Angreifermodell definiert die	maximal	berücksichtigte	Stärke eines Angreifers, gegen den ein	Schutzmechanismus	gerade noch wirkt.	
%Es beschreibt:
%–  Rollen des Angreifers (Außenstehender, Benutzer, Betreiber, Wartungsdienst, Produzent, Entwerfer …), auch kombiniert 
%–  Verbreitung des Angreifers (Stellen im System, an denen der Angreifer Informationen gewinnen oder Systemzustände verändern kann) 
%–  Verhalten des Angreifers 
%  •  passiv / aktiv,  beobachtend / verändernd 
%–  Rechenkapazität des Angreifers 
%  •  unbeschränkt: informationstheoretisch 
%  •  beschränkt: komplexitätstheoretisch 



\todo{Modell formulieren}

Ein Angreifermodell beschreibt die maximale Stärke eines Angreifers im Bezug auf verschiedene Faktoren, gegen die ein System abgesichert ist. Enthalten sind die Rolle eines Angreifers, die Verbreitung im System, aktives oder passives Verhalten und die Rechenkapazität, die der Angreifer zum Überwinden der schützenden Verfahren aufbringen kann. Es bildet die Basis für alle Folgeüberlegungen im Bezug auf die Sicherheit des zu entwickelnden Systems.

Logdaten kommen nicht-pseudonymisiert und unverschlüsselt über das Netzwerk (Nicht Fokus dieser Arbeit). Solange dies nicht geändert werden kann, liegt der Fokus auf sicherer Speicherung und geschütztem Zugriff auf die Daten in OSSIM.

- Prämisse: Das Pseudonym eines Nutzers erlaubt (ohne Anwendung von Hintergrundwissen) keinen Rückschluss auf die Identität eines Nutzers. Erst die Kollaboration ermöglicht das Aufdecken eines Pseudonyms.

- Rollen: Außenstehende, Benutzer/Administratoren mit OSSIM-Zugriff, Administratoren mit physischem Zugriff auf Rechner im Netz (auch P-Store)

- Verbreitung: Fokus liegt auf Speicherung, daher wird ungesicherte Übertragung von Logdaten, die vor der Pseudonymisierung stattfindet und in der Benutzerinformationen noch im Originalformat vorliegen, nicht betrachtet. Ein Ziel der Architektur ist natürlich trotzdem diesen Anteil/Weg möglichst gering zu halten.

- Verhalten: \todo{Aktive Angreifer?}

- Rechenkapazität: Verbreitete kryptographische Algorithmen werden als nicht mit vertretbarem Aufwand zu brechend angesehen, daher handelt es sich um die Annahme von komplexitätstheoretischer Sicherheit und einem in seiner Rechenleistung beschränkten Angreifer.






\subsection*{Anforderungen an die Integration in OSSIM}

Mögliche Eigenschaften, die gegeneinander abzuwägen sind:

- Abhängigkeit von der OSSIM-Konfiguration

- Erfordert Veränderung des SIEM-Systems

- Daten liegen temporär in nicht-pseudonymisierter Form in OSSIM vor

- Logdaten müssen mehrfach geparst werden


\subsection*{Anforderungen an die Pseudonymisierung}

- Lang genug für geringe Kollisionswsk.
- Eindeutig
- Durchsuchbar (mim Hinblick auf threshold)
- Anwendungsfallabhängige Parameter für Nutzzeit, ... (Rückblick auf Kapitel 3)

\subsection*{Anforderungen im Bezug auf den Einsatz eines kryptographischen Schwellwertschemas}

- Verteiltes Modell 
- Kommunikation
- Schlüsselmanagement
- ...



\subsection*{Erweiterbarkeit um neue Datenquellen}

Das umzusetzende System sollte es ermöglichen, Daten aus verschiedenen Quellen und (abhängig vom gewählten Eingriffspunkt in OSSIM auch) in verschiedenen Formaten entgegenzunehmen und mithilfe der umgesetzten Datenschutztechniken verändern zu können. Dabei muss das Format der Logdaten grundsätzlich beibehalten werden, um die Behandlung der Daten in OSSIM weiterhin zu ermöglichen.

\subsection*{Erweiterbarkeit um neue Datenschutztechniken}

Neben der im Fokus dieser Arbeit stehenden Pseudonymisierung und dem Einsatz von kryptographischen Schwellwertschemata zum Schutz der Logdaten gibt es weitere Datenschutztechniken, die für den Anwendungsfall genutzt werden könnten (siehe Kapitel \ref{cha_alternatives}). Der umgesetzte Prototyp sollte leicht um diese Techniken erweiterbar sein, d.h. so gestaltet sein, dass andere Techniken ohne große Änderungen am System integriert und auf eingehende Logdaten angewendet werden können.

