\chapter*{Aufgabenstellung}

Die technologiegestützte Bekämpfung von Insider-Angriffen im Unternehmenskontext basiert aktuell häufig auf der Analyse des Nutzerverhaltens einzelner Mitarbeiter und der Erkennung von Abweichungen zum erwarteten Normalverhalten. Diese sogenannte Anomalieerkennung benötigt umfassende Überwachungsdaten aller digitalen Endgeräte und Datenkommunikationssysteme zur Erstellung und eindeutigen Zuordnung von Nutzerprofilen. Dabei entsteht ein Konflikt mit dem Datenschutz der Mitarbeiter, da die Erhebung, Verarbeitung und Speicherung von Überwachungsdaten einen schweren Eingriff in die Privatsphäre und die informationelle Selbstbestimmung der Mitarbeiter darstellt. Um diesen Konflikt zu lösen, können auf der einen Seite Datenschutztechniken eingesetzt werden, die den unmittelbaren Personenbezug gesammelter Daten entfernen. Auf der anderen Seite kann mithílfe von Kryptographie die Rückgewinnung des Personenbezugs im Verdachtsfall und unter der Voraussetzung einer mehrseitigen Kollaboration ermöglicht werden.

Das Ziel der Masterarbeit ist die konzepzionelle Erarbeitung einer solchen datenschutzfreundlichen und mehrseitig sicheren Erhebung, Verarbeitung und Speicherung von Überwachungsdaten sowie die prototypische Implementierung auf Basis eines Security Information Event Management Systems. Dabei sollen insbesondere die folgenden Punkte bearbeitet werden:

\begin{itemize}
  \item Wie ist der aktuelle Stand sowohl der Technik als auch der Wissenschaft im Bereich der Pseudonymisierung und der kryptographischen Schwellwertschemata?
  \item An welcher Stelle des konzipierten Systems können die Überwachungsdaten entsprechend des Datenschutzes und der späteren möglicherweise erforderlichen Rückgewinnung des Personenbezugs verarbeitet werden und welche Auswirkungen können entstehen?
  \item Wie und an welcher Stelle muss das Schlüsselmanagement der benötigten kryptographischen Funktionen erfolgen?
  \item Welche Alternativen gibt es neben der Pseudonymisierung und den kryptographischen Schwellwertschemata zur Lösung des genannten Zielkonflikts und wie können diese in das Konzept und die prototypische Implementierung integriert werden?
\end{itemize}