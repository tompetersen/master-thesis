\chapter{Einführung}

\label{cha_introduction}

% Drei-Projekt erwähnen? Oder nur allgemeine Informationen entnehmen?



%- Insiderangriffe 

%- SIEM-Systeme in aktueller Form keine adäquate Lösung

%- Datenschutzrecht Arbeitnehmer

%- Mögliche Lösung: Pseudonymisierung und Schwellwertschemata mit verteilten Schlüsseln

%- Spannungsfeld Aufdeckbarkeit und Datenschutz

\todo{
Wo passen Abschnitte zu folgenden Stichworten hin?
- Verschiedene Datenarten: Identifizierend, Traffic, nicht relevant, ...
- Grundlegende Definition Insiderangriff
}

Liest man von erfolgreichen Angriffen auf Unternehmensnetzwerke, so ist die implizite Annahme von außenstehenden, unternehmensfremden Angreifern weit verbreitet. Doch häufig sind die Angreifer bereits im Netzwerk ansässig. Es handelt sich um (ehemalige) Mitarbeiter oder zumindest Personen mit legitimem Zugriff auf das Netzwerk, wie Geschäftspartnern oder Kunden. Hierbei geht es keineswegs lediglich um Einzelfälle. 

In dem \textit{IBM Cyber Security Intelligence Report} von 2015 werden 55\% der Angriffe als aus dem internen Netz stammend angegeben \cite{ibm2015}. Zu beachten ist allerdings, dass nicht nur mit Absicht ausgeführte Angriffe hierunter erfasst wurden, sondern auch unbeabsichtigte wie das versehentliche Veröffentlichen schützenswerter Kundendaten.

Auch der Branchenverband bitkom führt in seiner \textit{Spezialstudie Wirtschaftsschutz} aus dem Jahr 2016 nach einer Befragung von über 1000 Unternehmen aus, dass etwa 60\% der erfolgten Handlungen aus dem Bereich Datendiebstahl, Industriespionage oder Sabotage durch (ehemalige) Mitarbeiter erfolgten \cite{bitkom2016}.

\todo{Schadenshöhe (siehe Antrag)?}

Auch wenn die genauen Zahlen aufgrund von unterschiedlichen Annahmen und der in diesem Bereich nicht zu vernachlässigenden Dunkelziffer\footnote{
	Insbesondere die Angst vor Imageschäden, die auch in der \textit{Spezialstudie Wirtschaftsschutz} erwähnt wird, könnte ein Grund für das Geheimhalten von Vorfällen sein.
} mit Vorsicht zu betrachten sind, so geben sie doch Hinweise darauf, dass Angriffe von Innentätern weit verbreitet sind und ein hohes Schadenspotenzial aufweisen. Die Erkennung und Verhinderung solcher Angriffe sollte daher ein wichtiger Teil des IT-Sicherheitskonzepts eines Unternehmens sein.

Zur Erkennung von Angriffen in Netzwerken können SIEM-Systeme eingesetzt werden (siehe Abschnitt \ref{cha_siem}). Diese sind jedoch in erster Linie auf das Erkennen von externen Angriffen ausgelegt und in ihrer derzeitigen Form kaum sinnvoll für das Erkennen von Innentätern zu nutzen. \todo{EZ: Warum nicht? Einige werben damit, Innentaeter erkennen zu koennen, z.B. IBM QRadar} \\
Hierfür würden zusätzliche Datenquellen und Erkennungslogiken nötig sein.\todo{EZ: Warum? Welche?} 
Zusätzlich spielen auch  datenschutzrechtliche Bedenken im Bezug auf das Sammeln von großen Datenmengen über Mitarbeiter des eigenen Unternehmens hier eine entscheidende Rolle. Details hierzu sind im folgenden Kapitel \ref{cha_employee_privacy} zu finden.

Ein Ansatz, der diese Bedenken ausräumen oder zumindest lindern könnte, ist die Nutzung von Pseudonymen bei der Datenerfassung (siehe Abschnitt \ref{cha_pseudonym}). Anstatt direkt identifizierende Merkmale eines Arbeitnehmers abzuspeichern, werden diese Merkmale durch ein Pseudonym ersetzt. Eine Liste dieser Ersetzungen wird verschlüsselt abgelegt. Im Fall eines Angriffs durch einen Innentäter kann die Liste entschlüsselt werden und relevante Ereignisse de-pseudonymisiert, also ihrem ursprünglichen Verursacher wieder zweifelsfrei zugeordnet, werden.\\
Um die Entschlüsselung nicht einzelnen (möglicherweise bösartig agierenden) Personen zu ermöglichen, können sogenannte Schwellwertschemata eingesetzt werden (siehe Abschnitt \ref{cha_threshold}). Durch sie wird die Entschlüsselung erst durch die Kooperation mehrerer Parteien möglich gemacht.

Bei diesem Ansatz muss jedoch auch beachtet werden, dass durch den Einsatz von Pseudonymen die Erkennung von Angriffen erschwert werden könnte. Beispielsweise könnte das Ändern von Pseudonymen in regelmäßigen Zeitintervallen und die dadurch entstehende Nicht-Verkettbarkeit von Ereignissen dafür sorgen, dass längfristig angelegte Angriffe nicht aufgedeckt werden.

\section*{Ziele der Arbeit}

%- OSSIM: wo ansetzen? Agent, Client, dazwischen (eigene Komponente) Performancemessungen

%- Schlüsselmanagement (Clientseitig erzeugen, wie verteilen, etc.)

%- Welche kryptographischen Schwellwertschemata? Performancemessungen

%- Welche Funktionen? (Reine Verschlüsselung, Pseudonymisierung mit Mappingtabelle, ... -> erweiterbar)


In dieser Arbeit soll es darum gehen, prototypisch ein solches Szenario auf Basis eines Open-Source-SIEM-Systems umzusetzen.\todo{EZ: Was genau ist das Szenario? Liegt dein Fokus nun auf den zusaetzlichen Datenquellen und Erkennungslogiken oder auf den datenschutzrechtlichen Bedenken?} 
Hierbei müssen einige Fragen betrachtet werden:

\begin{itemize}
\item An welcher Stelle des Systems kann eingegriffen werden, um die erfassten Daten zu verändern, und welche Auswirkungen hat dies?
\item Wie erfolgt die angesprochene Pseudonymisierung technisch?
\item Welche kryptographischen Schwellwertschemata können genutzt werden? Gibt es bereits quelloffene Implementierungen? Was muss selbst entwickelt werden? Wie erfolgt das Schlüsselmanagement?
\item Können neben der Pseudonymisierung noch weitere Funktionen zur Veränderung von Daten sinnvoll sein und wie könnten diese umgesetzt werden?
\end{itemize}

Gerade die letzte Frage sorgt dafür, dass zusätzliche Anforderungen an den zu entwickelnden Prototypen gestellt werden. Es sollte möglich sein, abhängig von den eingehenden Daten die entsprechend gewünschten Funktionen konfigurieren und den Prototypen in eventuell aufbauenden Arbeiten auch um zusätzliche Funktionen ergänzen zu können.

\todo{Zu ergänzen: Aufbau der Arbeit, Literatur}


\section{Related work}
\label{related_work}

%\cite{salem2008survey}, A Survey of Insider Attack Detection Research, 2008, 195
%Übersicht über verschiedene Arten der Insider-Erkennung (Host-based and network-based), 
%we also believe that any technologies developed to detect insider attack have to include strong privacy-preserving guarantees,
%How might a system alert a supervisor of a possible attack without disclosing
%an employee’s true identity unless and until an attack has been validated?

Im Bereich der Erkennung von Insiderangriffen fand bereits einige Forschungsarbeit statt. In \cite{salem2008survey} bieten die Autoren einen Überblick über Forschungsergebnisse basierend auf unterschiedlichen Verfahren aus der Statistik und des maschinellen Lernens sowohl auf Host- als auch auf Netzwerkebene. Hier wird auch die Frage nach Erhalt der Privatsphäre eines Nutzers als Feld weiterer notwendiger Forschung dargestellt:

\begin{quotation}
Hence, we also believe that any technologies developed
to detect insider attack have to include strong privacy-preserving guarantees
to avoid making false claims that could harm the reputation of individuals
whenever errors occur. [...] 
How might a system alert a supervisor of a possible attack without disclosing
an employee’s true identity unless and until an attack has been validated?
\cite{salem2008survey}
\end{quotation}

%\cite{sobirey1997pseudonymous}, Pseudonymous Audit for Privacy Enhanced Intrusion Detection, 1997, 64
%Zwei Ansätze zur pseudonymen Intrusion Detection (IDA, AID), basieren auf symmetrischer Verschlüsselung (Erwähnung von 4-Augen-Prinzip durch Teilung des Schlüssels), Pseudonymisierung geschieht jeweils bereits im Betriebssystem-Kernel

%\cite{buschkes1999privacy}, Privacy enhanced intrusion detection, 1999, 23
%Architekturansatz unter Pseudonymnutzung, benötigt TTP für die Generierung und Aufdeckung von Pseudonymen, praktische Umsetzung mithilfe von Kerberos oder MIXen

%\cite{lundin1999privacy, lundin2000anomaly}, Privacy vs. Intrusion Detection Analysis, 1999, 35
%Anomalieerkennung unter Nutzung von Pseudonymisierung, Experimente mit Firewalldaten, Pseudonymaufdeckung als durch organisatorische Maßnahmen zu regelnder Prozess

%\cite{biskup2000threshold, biskup2001pseudonymization} On pseudonymization of audit data for intrusion detection, Threshold-based identity recovery for privacy enhanced applications, 2000, 44
%Pseudonymisierungsansatz, der Shares (Shamir's Secret Sharing) als Pseudonyme nutzt. Bei Überschreitung eines Schwellwerts kann so ein Verursacher, der in genug Verdachtsfällen auffiel, ermittelt werden.

Mit dieser Fragestellung beschäftigen sich weitere Veröffentlichungen im Bereich der Intrusion-Detection-Systeme. Oftmals wird -- wie in dieser Arbeit auch -- Pseudonymisierung als Verfahren zum Erhalt der Privatsphäre genutzt.\\
In \cite{sobirey1997pseudonymous} werden zwei Ansätze zur \textit{Privacy Enhanced Intrusion Detection} vorgestellt. Die Pseudonymisierung wird jeweils bereits im Betriebssystem-Kernel vorgenommen und mithilfe symmetrischer Verschlüsselung erreicht. Auch die Nutzung des 4-Augen-Prinzips wird bei der Pseudonymaufdeckung wird bereits erwähnt -- hier angestrebt durch eine Aufteilung des symmetrischen Schlüssels.\\
In \cite{buschkes1999privacy} stellt der Autor einen Architekturansatz für Intrusion-Detection-Systeme vor, der ebenfalls auf der Nutzung von Pseudonymen beruht. Es werden zwei Ansätze basierend auf Kerberos-Tickets bzw. dem MIX-Konzept vorgestellt. Für die Generierung bzw. Aufdeckung eines Pseudonyms wird eine \textit{Trusted Third Party} benötigt.\\
In \cite{lundin2000anomaly} wird von den Autoren ein System zur Anomalieerkennung auf Basis von Pseudonymen entwickelt und anhand von Logdaten einer Firewall überprüft. Das Aufdecken von Pseudonymen wird hier als durch organisatorische Maßnahmen zu regelnder Prozess verstanden.\\
In \cite{biskup2000threshold} und \cite{biskup2001pseudonymization} nutzen die Verfasser Shamir's Secret Sharing zur Erzeugung von Pseudonymen. Jeder Share bildet ein Pseudonym. Hierdurch wird sichergestellt, dass ein Pseudonym erst aufgedeckt werden kann, wenn eine einen Schwellwert übertreffende Anzahl von Warnmeldungen zu einem Nutzer im System eingetroffen ist.


%\cite{park2007ppids}, PPIDS : Privacy Preserving Intrusion Detection System, 2007, 7
%Rule-based pattern matching auf HIDS unter Nutzung von homomorpher Verschlüsselung ohne TTP, nur für bestimmte Situationen und relativ hoher Performanzoverhead

%\cite{niksefat2013zids}, ZIDS: A Privacy-Preserving Intrusion Detection System Using Secure Two-Party Computation Protocols, 2013, 11
%Client-Server Lösung zur Privacy-Preserving Intrusion Detection mit Geheimhaltungsgarantien für Clients (im Bezug auf zur Erkennung notwendige Daten) und Server (im Bezug auf Signaturen von Zero-Day-Exploits)

Neben den Pseudonym-basierten Lösungen gibt es weitere auf anderen Verfahren basierende Forschungsergebnisse. In \cite{park2007ppids} beispielsweise werden von den Autoren die Eigenschaften homomorpher Verschlüsselung zur Privatsphäre-erhaltenden Angriffserkennung eingesetzt, in \cite{niksefat2013zids} wird eine Art der Mehrparteienberechnung genutzt, um mehrseitigen Datenschutz insbesondere im Hinblick auf Zero-Day-Lücken in einem Intrusion-Detection-System zu garantieren.\\
In \cite{niksefat2017privacy} bieten die Autoren einen Überblick über weitere existierende Lösungen im Bereich der \textit{Privacy Enhanced Intrusion Detection}.

%\cite{niksefat2017privacy}, Privacy issues in intrusion detection systems: A taxonomy, survey and future directions, 2017, 0
%Survey über existierende Lösungen im Bereich der Privacy preserving intrusion detection, für Kapitel Alternative Datenschutztechniken auch Übersicht über Privacy preserving techniques in IDS.