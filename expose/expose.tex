%!TEX encoding = UTF-8 Unicode
\documentclass[
    fontsize=12pt,
    headings=small,
    parskip=half,           % Ersetzt manuelles setzten von parskip/parindent.
    bibliography=totoc,
    numbers=noenddot,       % Entfernt den letzten Punkt der Kapitelnummern.
    open=any,               % Kapitel kann auf jeder Seite beginnen.
%   final                   % Entfernt alle todonotes und den Entwurfstempel.
    ]{scrreprt}

% ===================================Praeambel==================================

% Kodierung, Sprache, Patches {{{
\usepackage[T1]{fontenc}    % Ausgabekodierung; ermoeglicht Akzente und Umlaute
                            %  sowie korrekte Silbentrennung.
\usepackage[utf8]{inputenc} % Erlaub die direkte Eingabe spezieller Zeichen.
                            %  Utf8 muss die Eingabekodierung des Editors sein.
\usepackage[ngerman]{babel} % Deutsche Sprachanpassungen (z.B. Ueberschriften).
\usepackage{microtype}      % Optimale Randausrichtung und Skalierung.
\usepackage[
    autostyle,
    ]{csquotes}             % Korrekte Anfuehrungszeichen in der Literaturliste.
\usepackage{fixltx2e}       % Patches fuer LaTeX2e.
\usepackage{scrhack}        % Verhindert Warnungen mit aelteren Paketen.
\usepackage[
  newcommands
]{ragged2e}                 % Verbesserte \ragged...Befehle
\PassOptionsToPackage{
  hyphens
}{url}                      % Sorgt für URL-Umbrueche in Fusszeilen u. Literatur
% }}}

% Schriftarten {{{
\usepackage{mathptmx}       % Times; modifies the default serif and math fonts
\usepackage[scaled=.92]{helvet}% modifies the sans serif font
\usepackage{courier}        % modifies the monospace font
% }}}

% Biblatex {{{
\usepackage[
    style=alphabetic,
    backend=bibtex,
    %backref=true
    ]{biblatex}             % Biblatex mit alphabetischem Style und biber.
\bibliography{literature}      % Dateiname der bib-Datei.
%\addbibresource{literature.bib}
\DeclareFieldFormat*{title}{
    \mkbibemph{#1}}         % Make titles italics
% }}}

% Dokument- und Texteinstellungen {{{
\usepackage[
    a4paper,
    margin=2.54cm,
    marginparwidth=2.0cm,
    footskip=1.0cm
    ]{geometry}             % Ersetzt 'a4wide'.
\clubpenalty=10000          % Keine Einzelzeile am Beginn eines Paragraphen
                            %  (Schusterjungen).
\widowpenalty=10000         % Keine Einzelzeile am Ende eines Paragraphen
\displaywidowpenalty=10000  %  (Hurenkinder).
\usepackage{floatrow}       % Zentriert alle Floats.
\usepackage{ifdraft}        % Ermoeglicht \ifoptionfinal{true}{false}
\pagestyle{plain}           % keine Kopfzeilen
% \sloppy                    % großzügige Formatierungsweise
\deffootnote{1em}{1em}{
  \thefootnotemark.\ }      % Verbessert Layout mehrzeiliger Fußnoten

\makeatletter
\AtBeginDocument{%
    \hypersetup{%
        pdftitle = {\@title},
        pdfauthor  = \@author,
    }
}
\makeatother
% }}}

% Weitere Pakete {{{
\usepackage{graphicx}       % Einfuegen von Graphiken.
\usepackage{tabu}           % Einfuegen von Tabellen.
\usepackage{multirow}       % Tabellenzeilen zusammenfassen.
\usepackage{multicol}       % Tabellenspalten zusammenfassen.
\usepackage{booktabs}       % Schönere Tabellen (\toprule\midrule\bottomrule).
\usepackage[nocut]{thmbox}  % Theorembox bspw. fuer Angreifermodell.
\usepackage{amsmath}        % Erweiterte Handhabung mathematischer Formeln.
\usepackage{amssymb}        % Erweiterte mathematische Symbole.
\usepackage{rotating}
\usepackage[
    printonlyused
    ]{acronym}              % Abkuerzungsverzeichnis.
\usepackage[
    colorinlistoftodos,
    textsize=tiny,          % Notizen und TODOs - mit der todonotes.sty von
    \ifoptionfinal{disable}{}%  Benjamin Kellermann ist das Package "changebar"
    ]{todonotes}            %  bereits integriert.
\usepackage[
    breaklinks,
    hidelinks,
    pdfdisplaydoctitle,
    pdfpagemode = {UseOutlines},
    pdfpagelabels,
    ]{hyperref}             % Sprungmarken im PDF. Laed das URL Paket.
    \urlstyle{rm}           % Entfernt die Formattierung von URLs.
%\usepackage{breakurl}
%\def\UrlBreaks{\do\/\do-}
\usepackage{listings}       % Spezielle Umgebung für...
    \lstset{                %  ...Quelltextformatierung.
        language=C,
        breaklines=true,
        breakatwhitespace=true,
        frame=L,
        captionpos=b,
        xleftmargin=6ex,
        tabsize=4,
        numbers=left,
        numberstyle=\ttfamily\footnotesize,
        basicstyle=\ttfamily\footnotesize,
        keywordstyle=\bfseries\color{green!50!black},
        commentstyle=\itshape\color{magenta!90!black},
        identifierstyle=\ttfamily,
        stringstyle=\color{orange!90!black},
        showstringspaces=false,
        }

% ===================================Dokument===================================

\title{Exposé zur Masterarbeit - Pseudonymisierung von und Einsatz von Schwellwertschemata für Logeinträge}
\author{Tom Petersen}
% \date{01.01.2015} % falls ein bestimmter Tag eingesetzt werden soll, einfach
                    %  diese Zeile aktivieren

\begin{document}

% ================================Deckblatt-Muster==============================
\newpage
\thispagestyle{empty}
% \addcontentsline{toc}{chapter}{Muster des Deckblatts}
\begin{titlepage}% {{{
\includegraphics[width=6.8cm]{./img/up-uhh-logo-u-2010-u-farbe-u-rgb.pdf}
\begin{center}\Large
	% Universität Hamburg \par
	% Fachbereich Informatik
	\vfill
	Masterarbeit
	\vfill
	\makeatletter
	{\Large\textsf{\textbf{\@title}}\par}
	\makeatother
	\vfill
	vorgelegt von
	\par\bigskip
	\makeatletter
	{\@author} \par
	\makeatother
	geb. am 13. Dezember 1990 in Hannover\par
	Matrikelnummer 3659640 \par
	Studiengang Informatik
	\vfill
	\makeatletter
	eingereicht am {\@date}
	\makeatother
	\vfill
	Betreuer: Dipl.-Inf. Ephraim Zimmer\par
	Erstgutachter: - \par
	Zweitgutachter: -
\end{center}
\ifoptionfinal{}{
	\begin{tikzpicture}[remember picture, overlay]
	    \node[draw, red, font=\ttfamily\bfseries\Huge, xshift=50mm, yshift=228mm,
	        rotate=340, text centered, text width=8cm, very thick, rounded
	        corners=4mm] at (current page.south) {Entwurf vom \today};
	\end{tikzpicture}}
\end{titlepage}% }}}

% ================================Content==============================

% Drei-Projekt erwähnen? Oder nur allgemeine Informationen entnehmen?

\chapter{Einführung}

Im Folgenden soll zuerst das Thema der Arbeit motiviert und anschließend auf mögliche Schwerpunkte, die für die Arbeit gesetzt werden können, eingegangen werden.

\section{Motivation}

%- Insiderangriffe 

%- SIEM-Systeme in aktueller Form keine adäquate Lösung

%- Datenschutzrecht Arbeitnehmer

%- Mögliche Lösung: Pseudonymisierung und Schwellwertschemata mit verteilten Schlüsseln

%- Spannungsfeld Aufdeckbarkeit und Datenschutz



Liest man von erfolgreichen Angriffen auf Unternehmensnetzwerke, so ist die implizite Annahme von außenstehenden, unternehmensfremden Angreifern weit verbreitet. Doch häufig sind die Angreifer bereits im Netzwerk unterwegs. Es handelt sich um (ehemalige) Mitarbeiter oder zumindest Personen mit legitimem Zugriff auf das Netzwerk, wie Geschäftspartnern oder Kunden. Und dies sind keine Einzelfälle. 

In dem \textit{IBM Cyber Security Intelligence Report} von 2015 werden 55\% der Angriffe als aus dem internen Netz stammend angegeben \cite{ibm2015}. Zu beachten ist, dass nicht nur mit Absicht ausgeführte Angriffe hierunter erfasst wurden, sondern auch unbeabsichtigte wie das versehentliche Veröffentlichen schützenswerter Kundendaten.

Auch der Branchenverband bitkom führt in seiner \textit{Spezialstudie Wirtschaftsschutz} aus dem Jahr 2016 nach einer Befragung von über 1000 Unternehmen aus, dass etwa 60\% der erfolgten Handlungen aus dem Bereich Datendiebstahl, Industriespionage oder Sabotage durch (ehemalige) Mitarbeiter erfolgten \cite{bitkom2016}.

\todo{Schadenshöhe (siehe Antrag)?}

Auch wenn die genauen Zahlen aufgrund von unterschiedlichen Annahmen und der in diesem Bereich nicht zu vernachlässigenden Dunkelziffer\footnote{Insbesondere die Angst vor Imageschäden, die auch in der \textit{Spezialstudie Wirtschaftsschutz} erwähnt wird, könnte ein Grund für das Geheimhalten von Vorfällen sein.} mit Vorsicht zu betrachten sind, so zeigen sie doch, dass Angriffe von Innentätern weit verbreitet sind und ein hohes Schadenspotenzial aufweisen. Die Erkennung und Verhinderung solcher Angriffe sollte daher ein wichtiger Teil des IT-Sicherheitskonzepts eines Unternehmens sein.

Zur Erkennung von Angriffen in Netzwerken werden häufig SIEM-Systeme eingesetzt (siehe Abschnitt \ref{sec_siem}). Diese sind jedoch in erster Linie auf das Erkennen von externen Angriffen ausgelegt und in ihrer derzeitigen Form kaum sinnvoll für das Erkennen von Innentätern zu nutzen. 

Hierfür würden zusätzliche Datenquellen und Erkennungslogiken nötig sein. Zusätzlich spielen auch  datenschutzrechtliche Bedenken im Bezug auf das Sammeln von großen Datenmengen über Mitarbeiter des eigenen Unternehmens hier eine entscheidende Rolle. \todo{Beispiel für Datenschutz für Arbeitnehmer}\\
Ein Ansatz, der diese Bedenken ausräumen oder zumindest lindern könnte, ist die Nutzung von Pseudonymen bei der Datenerfassung. Anstatt direkt identifizierende Merkmale eines Arbeitnehmers abzuspeichern, werden diese Merkmale durch ein Pseudonym ersetzt. Eine Liste dieser Ersetzungen wird verschlüsselt abgelegt. Im Fall eines Angriffs durch einen Innentäter kann die Liste entschlüsselt werden und relevante Ereignisse de-pseudonymisiert, also ihrem ursprünglichen Verursacher wieder zweifelsfrei zugeordnet, werden.\\
Um die Entschlüsselung nicht einzelnen (möglicherweise auch bösartigen) Personen zu ermöglichen, können sogenannte Schwellwertschemata eingesetzt werden (siehe Abschnitt \ref{sec_threshold}). Durch sie wird die Entschlüsselung erst durch die Kooperation mehrerer Parteien möglich gemacht.

Bei diesem Ansatz muss jedoch auch beachtet werden, dass durch den Einsatz von Pseudonymen die Erkennung von Angriffen erschwert werden könnte. Beispielsweise könnte das Ändern von Pseudonymen in regelmäßigen Zeitintervallen und die dadurch entstehende Nicht-Verkettbarkeit von Ereignissen dafür sorgen, dass längfristig angelegte Angriffe nicht aufgedeckt werden.

\section{Ziele der Arbeit}

%- OSSIM: wo ansetzen? Agent, Client, dazwischen (eigene Komponente) Performancemessungen

%- Schlüsselmanagement (Clientseitig erzeugen, wie verteilen, etc.)

%- Welche kryptographischen Schwellwertschemata? Performancemessungen

%- Welche Funktionen? (Reine Verschlüsselung, Pseudonymisierung mit Mappingtabelle, ... -> erweiterbar)


In dieser Arbeit soll es darum gehen, prototypisch ein solches Szenario auf Basis eines Open-Source-SIEM-Systems umzusetzen. Hierbei müssen einige Fragen betrachtet werden:

\begin{itemize}
\item An welcher Stelle des Systems kann eingegriffen werden, um die erfassten Daten zu verändern, und welche Auswirkungen hat dies?
\item Wie erfolgt die angesprochene Pseudonymisierung technisch?
\item Welche kryptographischen Schwellwertschemata können genutzt werden? Gibt es bereits quelloffene Implementierungen? Was muss selbst entwickelt werden? Wie erfolgt das Schlüsselmanagement?
\item Können neben der Pseudonymisierung noch weitere Funktionen sinnvoll sein und wie könnten diese umgesetzt werden?
\end{itemize}

Gerade die letzte Frage sorgt dafür, dass zusätzliche Anforderungen an den zu entwickelnden Prototypen gestellt werden. Es sollte möglich sein, abhängig von den eingehenden Daten die entsprechend gewünschten Funktionen konfigurieren und den Prototypen in aufbauenden Arbeiten auch um zusätzliche Funktionen ergänzen zu können.


\chapter{Inhalte}

\section{SIEM-Systeme}

\label{sec_siem}

- Was ist das?

- OSSIM als Open-Source-Vertreter

%\section{Pseudonymisierung}

%Pseudonymisierung als Möglichkeit der Verschleierung und Nicht-Verkettbarkeit.

\section{Schwellwertschemata}

\label{sec_threshold}

- Shamir How to share a secret?

- Public Key Problematik

- Was ist das? (siehe auch Paper für Definition)

- Fünde (RSA, Paillier, ...) und Desmedt/Frankel evtl. hier schon Pedersen/...

% ================================Literature==============================

\begin{raggedright}         % Schaltet Blocksatz ab, erzeugt ein stimmigeres
                            %  Schriftbild im Literaturverzeichnis.
  \printbibliography        % Falls Biblatex verwendet wird.
  \label{sec:literaturverzeichnis}
\end{raggedright}

\end{document}

